% Minimal Working Example
% This about the minimum you need
% to type to have a working LaTeX
% file. Typesets a basic article with
% a title, author and date.
% Adding structure and indices, EPIGRAPHS
% Example-06

\documentclass{book}
\usepackage{microtype}
\usepackage{lipsum}
\usepackage{makeidx}
\usepackage{epigraph}
\usepackage[listings,theorems]{tcolorbox}
\tcbset{before={\par\medskip\pagebreak[0]\noindent},after={\par\medskip}}%
\newcounter{texercise}[section]
\makeindex

\title{My First Book}
\author{Y Lazarides}

\begin{document}

\maketitle
\tableofcontents

\chapter{Maths and more on Packages}

\section{Table of Equations}


\subsection{Display Style}

 \[\sum\limits_{i=1}^n i = \frac{n(n+1)}{2}\]


You can typeset them nicely in boxes using the \texttt{tcolorbox}. 
package.

\begin{tcolorbox}[colback=blue!5,
                  colframe=blue!50!black,
                  arc=0mm,
                  theorem={Equation}{texercise}{Summation}{myMarker}]{Summation of Numbers}{}
  For all natural number $n$ it holds:\\[2mm]
  \[\displaystyle\sum\limits_{i=1}^n i = \frac{n(n+1)}{2}\]
\end{tcolorbox}


\chapter{The Social Life of Rabbits}
\epigraph{Oh!  My ears and whiskers!}%
         {Lewis Carroll}


\section{The Epigraph Package}
\lipsum[1-30]
\section{More Sections}
\lipsum



\section{Another Section}
\lipsum
\appendix
\chapter{Experimental Results}
We now write about a book on science. Indexing is an
art and a science. This is our first example on indexing.\index{Indexing}



If there is one issue that gives me problem generally with windows, is
configuring a new installation properly. First port of call is normally
to set the path. I can assure you when DOS was around it was a smoother
process. We used to have get it on an autoexec.bat file and edit it with
a text editor. Now you have to first find the screen, which has been designed
with ultimate usability in mind. It normally puts a tonne of text in a tiny
little editor.



























\printindex
\end{document}