\documentclass{article}
\usepackage{amsmath,fp}
\begin{document}
\makeatletter
\count@=1
\def\DecimalToFraction#1{
%helper macro
\FPset\zero{0}
\FPset\X{#1}

%% Set initial values
\FPadd\X{\X}{0.0000000001} % avoid overflows and divisions by zero
\FPset\Zi{\X}
\FPset\Di{1}
\FPset\Dprevious{0}
%% begin loop
\loop\ifnum\count@<13
%% numerator term
\FPtrunc\temp{\Zi}{0} 
\FPsub\temp{\Zi}{\temp}
%% inverse
\FPdiv\Znext{1}{\temp}
%% Find Dnext
\FPtrunc\IntZnext{\Znext}{0}
%% Di x Int{Zi+1}
\FPmul\temp{\Di}{\IntZnext}
\FPadd\temp{\Di}{\Dprevious}
\FPset\Dnext{\temp}
\FPround\Dnext{\Dnext}{0}

%%% Find Ni+1
\FPmul\temp{\X}{\Dnext}
\FPround\temp{\temp}{0}
\FPset\Nnext{\temp}

\FPdiv\ratio{\Nnext}{\Dnext}

\(Z_i=\Znext\to \Nnext/\Dnext =\ratio\)

\FPset{\Dprevious}{\Dnext}
\FPset{\Di}{\Dprevious}
\FPset{\Zi}{\Znext}

\advance\count@ by1
\repeat
%% end of loop

\gdef\NUM{\Nnext}
\gdef\DEN{\Dnext}

\makeatother
}

\def\Test#1{%
  \DecimalToFraction{#1}
  %The number $#1=\frac{\NUM}{\DEN}$
}

\Test{.313}
\end{document}

















