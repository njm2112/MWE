%% A LaTeX MWE
%% myage.tex: Code for calculating one's age
%% given the date of birth
%% -------------------------------------------------------------------------------------------
%% Copyright (c) 2012 by  Dr. Yiannis Lazarides <yannislaz@gmail.com >
%% -------------------------------------------------------------------------------------------
%%
%% This work may be distributed and/or modified under the
%% conditions of the LaTeX Project Public License, either version 1.3
%% of this license or (at your option) any later version.
%% The latest version of this license is in
%%   http://www.latex-project.org/lppl.txt
%% and version 1.3 or later is part of all distributions of LaTeX
%% version 2005/12/01 or later.
%%
%% This work has the LPPL maintenance status `author-maintained'.
%%
%% This work consists of only the file myage.tex
%% To use 
\documentclass{article}
\usepackage{datenumber,fp}
\begin{document}
% We use two counters to store today's date and the birthday.

\newcounter{dateone}%
\newcounter{datetwo}%

% We set them at the respective dates
\setmydatenumber{dateone}{1989}{08}{01}%
\setmydatenumber{datetwo}{\the\year}{\the\month}{\the\day}%

%We then use the FP package to calculate them. The use of
% 365.25 to average the year is arguable but to one decimal
% place is immaterial

\FPsub\result{\thedatetwo}{\thedateone}
\FPdiv\myage{\result}{365.25} 

% Round the date to one decimal place.
\FPround\myage{\myage}{1}\myage\ years old

\end{document}