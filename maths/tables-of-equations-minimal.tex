% MWE to demonstrate inclusion of a Table of Formulae
% The package sethyprref
% is uded to provide the relevant settings for screen viewing
% it shrinks the screen size. 
% Dr Yiannis Lazarides 2012
% Version 1.0
\documentclass{book}
\usepackage{longtable,fancyhdr,xcolor,lipsum}
\usepackage{amsmath}[2000/07/18] 
\usepackage{amssymb}[2002/01/22] 
\newcommand{\DittoMark}{``}%''
\newcommand{\Ditto}{\quad\DittoMark\quad}
\newcommand{\Headings}[1]{\textbf{\small#1}}
\usepackage{sethyperref}
\begin{document}
\setcounter{secnumdepth}{-3}
\tableofcontents
\part*{First part}
\makeatletter


\makeatother
\chapter{General}
\lipsum[1-2]

\label{aref}%
 \[e= \frac{2}{a}
             \sqrt{s(s-a)(s-b)(s-c)} \] 

\chapter{TABLE OF FORMULAS.}
\markboth{\Headings{PLANE GEOMETRY.}}{\Headings{TABLE OF FORMULAS.}}%

\subsection{PLANE FIGURES.}

\subsection{NOTATION.}

\begin{tabular}{r@{~}c@{~}l}
$P$ &=& perimeter. \\
$h$ &=& altitude. \\
$b$ &=& lower base. \\
$b'$ &=& upper base. \\
$R$ &=& radius of circle. \\
$D$ &=& diameter of circle. \\
$C$ &=& circumference of circle. \\
$r$ &=& apothem of regular polygon. \\
$a$, $b$, $c$ &=& sides of triangle. \\
$s$ &=& \( \frac{1}{2}(a+b+c) \). \\
$p$ &=& perpendicular of triangle. \\
$m,n$ &=& segments of third side of triangle adjacent to \\
&& sides $b$ and $a$, respectively. \\
$S$ &=& area. \\
$\pi$ &=& 3.1416.
\end{tabular}

\newpage
\subsection{FORMULAS.}

\noindent\begin{longtable}{lr@{~}c@{~}l@{\qquad}r}
\multicolumn{5}{l}{\hspace{-2ex}\textbf{Line Values.}} \\
\multicolumn{5}{r}{\tiny PAGE}\\
\multicolumn{4}{l}{Altitude of triangle on side $a$,} \\
& $h$ &=& \( \displaystyle \frac{2}{a}
             \sqrt{s(s-a)(s-b)(s-c)} \) & \pageref{aref} \\
\multicolumn{4}{l}{Median of triangle on side $a$,} \\
& $m$ &=& \( \frac{1}{2} \sqrt{2(b^2+c^2) - a^2} \) & \pageref{aref} \\
\multicolumn{5}{l}{\hspace{-2ex}\textbf{Areas.}} \\
Rectangle,     & $S$ &=& $b\times h$ & \pageref{aref} \\
Square,        & $S$ &=& $b^2$ & \pageref{aref} \\
\end{longtable}


\appendix
\chapter{Experimental Data}
\section{First section}
\lipsum[1]
\subsection{First subsection}
\lipsum[1]
\chapter{Experimental Errors}
\section{First section}
\lipsum[1]
\subsection{First subsection}
\lipsum[1]
\end{document}
