\documentclass{article}
\begin{document}

\def\frac#1#2{{\begingroup#1\endgroup\over#2}}

\[ \frac12 \]

\def\frac#1#2{{#1\over#2}}

$\frac{a+23+c+2}{24}$


\def\startlines{%
  \begingroup
  \obeylines\obeyspaces
  \leavevmode
}
\def\endlines{\endgroup}

\startlines

This is a test 
...   ......         ....... test and another test


This is a test 
...   ......         ....... test and another test

\endlines

\obeylines\obeyspaces

  This is a test 
  ...   ......        
  ....... test and another test


\def\test#1{\bgroup#1\egroup}

$a=a+123-223\test{ }+z$

\def\test#1{\begingroup#1\endgroup}

$a=a+123-223\test{ }+z$


\makeatletter
\long\def\g@addto@macro#1#2{%
\begingroup
  \toks@\expandafter{#1#2}%
  \xdef#1{\the\toks@}%
\endgroup}

\def\test{}

\g@addto@macro{\test}{Write something. \def\A{Test}}
\def\A{28}
\g@addto@macro{\test}{Write something else \A }

\test


\def\alist{\@elt a\@elt b\@elt c}

\begingroup
  \let \@elt\@gobble  \alist
\endgroup

\begingroup
  \def\@elt#1{-#1- }  \alist
\endgroup

\makeatother


\end{document}

The use of groups follows common practices in computer coding. Here are some
common use cases.


In \@elt lists inevitably most changes should be local

When changing catcodes


When macro definitions are split into two


Text typesetting in a limited scope 





We use \begingroup
for this purpose since \define@newfont might be called in math mode, and an
empty \bgroup. . . \egroup would add an empty Ord atom to the math list and
thus affect the spacing.

to avoid `leaking out' of style changes

\let\math@bgroup\bgroup
448 \def\math@egroup#1{#1\egroup}


\def\index{\@bsphack\begingroup \@sanitize\@index}

\def\@index#1{\endgroup\@esphack}

\def\ProvidesFile#1{%
103 \begingroup
104 \catcode`\ 10 %
105 \ifnum \endlinechar<256 %
106 \ifnum \endlinechar>\m@ne
107 \catcode\endlinechar 10 %
108 \fi
109 \fi
110 \@makeother\/%
111 \@makeother\&%
112 \kernel@ifnextchar[{\@providesfile{#1}}{\@providesfile{#1}[]}}
During initex a special version of \@providesfile is used. The real denition
is installed right at the end, in ltfinal.dtx.
\def\@providesfile#1[#2]{%
\wlog{File: #1 #2}%
\expandafter\xdef\csname ver@#1\endcsname{#2}%
\endgroup}
\end{macrocode}

