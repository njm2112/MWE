% !TEX TS-program = pdflatex
% !TEX encoding = UTF-8 Unicode

% Example of the Memoir class, an alternative to the default LaTeX classes such as article and book, with many added features built into the class itself.

\documentclass[a4paper, oneside, justified=true, sfsidenotes]{tufte-book} % for a long document
%\documentclass[12pt,a4paper, showtrims]{memoir} % for a short document

\usepackage[utf8]{inputenc} % set input encoding to utf8
\usepackage[greek, french,english]{babel}[2005/11/23]
\usepackage{lipsum}
\usepackage{listings}
\usepackage{lineno}
\usepackage{xcolor}
\usepackage{gantt}
\usepackage{booktabs}
\usepackage{goodlists}
%% Bibliography
\usepackage{natbib}

\usepackage{shapepar}


\bibliographystyle{plainnat} % note the change here
\bibpunct{(}{)}{;}{a}{,}{,}


\usepackage{amsmath}[2000/07/18] %% Displayed equations
\usepackage{amssymb}[2002/01/22] %% and additional symbols

\usepackage{alltt}[1997/06/16]   %% boilerplate, credits, license
\usepackage{epigraph}
% use courier acrobat fonts rather than the type writer font
%\usepackage{courier}
\usepackage{shapepar}
\usepackage{lettrine}
\usepackage{graphicx}

%% Use some archaic fonts for fun
\usepackage{cypriot}
\usepackage{hieroglf}
\usepackage{verbatim}
\usepackage{caption}
\DeclareCaptionFont{blue}{\color{blue}}
\captionsetup{justification=raggedright, singlelinecheck=false,font={blue,sf,small}}


\usepgflibrary{shapes.symbols}
\usetikzlibrary{positioning, shapes, arrows, chains}
\usepackage{pgfplots}
%\pgfplotsset{compat=1.3}




%overwrite tufte and have numbering
\setcounter{secnumdepth}{5}


%% Change some of the looks
%%% CHAPTER STYLES AND SECTION STYLES
\pagestyle{fancy}
\renewcommand{\chaptermark}[1]%
{\markboth{\MakeUppercase{\chaptertitlename\ \thechapter\ #1}}{}} %no dot here
\renewcommand{\sectionmark}[1]%
{\markright{\MakeUppercase{\thesection~\ #1}}}
\renewcommand{\headrulewidth}{0.5pt}
\renewcommand{\footrulewidth}{0pt}
\newcommand{\helv}{%
\fontfamily{phv}\fontseries{b}\fontsize{9}{11}\selectfont}
\fancyhf{}
\fancyhead[LE,RO]{\helv \thepage}
\fancyhead[LO]{\helv \rightmark}
\fancyhead[RE]{\helv \leftmark}

%%hyperref
\hypersetup{urlcolor=green, colorlinks=true}  % Colours hyperlinks in blue, but this can be distracting if there are many links
\urlstyle{sf}  %set url as sans-serif font (in url? find how to do with hyperref)





% Typesets the font size, leading, and measure in the form of 10/12x26 pc.
\newcommand{\measure}[3]{#1/#2$\times$\unit[#3]{pc}}

% Macros for typesetting the documentation

\newcommand{\fox}{\medskip  "The quick brown fox jumps over the lazy dog"\medskip} % mini lorem
\newcommand{\dogs}{The sentence is often mistakenly rendered as "The quick brown fox jumped over the lazy dog," which does not include an s. However, this can be corrected by typing: "The quick brown fox jumped over the lazy dogs".}


\newcommand{\hlred}[1]{\textcolor{Maroon}{#1}}% prints in red

\newcommand{\hlmaroon}[1]{\textcolor{Maroon}{#1}}%print maroon

\newcommand{\hangleft}[1]{\makebox[0pt][r]{#1}}


\newcommand{\sourceatright}[2]{{%
  \unskip\nobreak\hfil\penalty100
  \hskip#1\hbox{}\nobreak\hfil{#2}
  \parfillskip 15pt \par}}


\newcommand{\hairsp}{\hspace{1pt}}% hair space
\newcommand{\hquad}{\hskip0.5em\relax}% half quad space
\newcommand{\TODO}{\textcolor{red}{\bf TODO!}\xspace}

\newcommand{\ie}{\textit{i.\hairsp{}e.}\xspace}
\newcommand{\eg}{\textit{e.\hairsp{}g.}\xspace}


%\newcommand{\na}{\quad--}% used in tables for N/A cells

\providecommand{\XeLaTeX}{X\lower.5ex\hbox{\kern-0.15em\reflectbox{E}}\kern-0.1em\LaTeX}
\newcommand{\tXeLaTeX}{\XeLaTeX\index{XeLaTeX@\protect\XeLaTeX}}
% \index{\texttt{\textbackslash xyz}@\hangleft{\texttt{\textbackslash}}\texttt{xyz}}

%% define backsalsh
%% you can also use \textbackslash
\newcommand{\tuftebs}{\symbol{'134}}% a backslash in tt type in OT1/T1

\newcommand{\doccmdnoindex}[2][]{\texttt{\tuftebs#2}}% command name -- adds backslash automatically (and doesn't add cmd to the index)

\newcommand{\doccmddef}[2][]{%
  \hlred{\texttt{\tuftebs#2}}\label{cmd:#2}%
  \ifthenelse{\isempty{#1}}%
    {% add the command to the index
      \index{#2 command@\protect\hangleft{\texttt{\tuftebs}}\texttt{#2}}% command name
    }%
    {% add the command and package to the index
      \index{#2 command@\protect\hangleft{\texttt{\tuftebs}}\texttt{#2} (\texttt{#1} package)}% command name
      \index{#1 package@\texttt{#1} package}\index{packages!#1@\texttt{#1}}% package name
    }%
}% command name -- adds backslash automatically

%% doc commands
% % adds hem automatically to index

\newcommand{\doccmd}[2][]{%
  \texttt{\hlred{\tuftebs#2}}%
    \ifthenelse{\isempty{#1}}%
    {% add the command to the index
      \index{#2 command@\protect\hangleft{\texttt{\tuftebs}}\texttt{#2}}% command name
      % \marginpar{\hlred{#1}}
    }%
    {% add the command and package to the index
      \index{#2 command@\protect\hangleft{\texttt{\tuftebs}}\texttt{#2} (\texttt{#1} package)}% command name
     \index{#1 package@\texttt{#1} package}\index{packages!#1@\texttt{#1}}% package name
    }%
}% command name -- adds backslash automatically



%% doc options
%
\newcommand{\docopt}[1]{\ensuremath{\langle}\textrm{\textit{#1}}\ensuremath{\rangle}}% optional command argument


\newcommand{\docarg}[1]{\textrm{\textit{#1}}}% (required) command argument
\newenvironment{docspec}{\begin{quotation}\ttfamily\parskip0pt\parindent0pt\ignorespaces}{\end{quotation}}% command specification environment


\newcommand{\docenv}[1]{\texttt{#1}\index{#1 environment@\texttt{#1} environment}\index{environments!#1@\texttt{#1}}}% environment name


\newcommand{\docenvdef}[1]{\hlred{\texttt{#1}}\label{env:#1}\index{#1 environment@\texttt{#1} environment}\index{environments!#1@\texttt{#1}}}% environment name

%% Provide a  command to add packages to index
%
\newcommand{\docpkg}[1]{\hlred{\texttt{#1}}\index{#1 package@\texttt{#1} package}\index{packages!#1@\texttt{#1}}
\marginpar{\hlred{#1~package}}
}% package name

%% Provide a command for document cls - do not add to index
%
\newcommand{\doccls}[1]{\texttt{#1}}% document class name


\newcommand{\docclsopt}[1]{\texttt{#1}\index{#1 class option@\texttt{#1} class option}\index{class options!#1@\texttt{#1}}}% document class option name
\newcommand{\docclsoptdef}[1]{\hlred{\texttt{#1}}\label{clsopt:#1}\index{#1 class option@\texttt{#1} class option}\index{class options!#1@\texttt{#1}}}% document class option name defined
\newcommand{\docmsg}[2]{\bigskip\begin{fullwidth}\noindent\ttfamily#1\end{fullwidth}\medskip\par\noindent#2}
\newcommand{\docfilehook}[2]{\texttt{#1}\index{file hooks!#2}\index{#1@\texttt{#1}}}
\newcommand{\doccounter}[1]{\texttt{#1}\index{#1 counter@\texttt{#1} counter}}

%% margin document commands
%
\newcommand{\margindoc}[1]{\marginpar[]{\doccmd{#1}}}

\newcommand{\sidebarcolor}[1]{\textcolor{darkgray}{\sidenote[]{#1}}}

%% alias of \doccmnd
\newcommand{\cmd}[1]{\doccmd{#1} \marginpar[]{\hlred{\doccmd{#1}}}}


%% Set up the epigraph to be a bit wider
\setlength{\epigraphwidth}{8cm} 
\setlength{\epigraphrule}{0pt}
\newcommand{\theepigraph}[2]{\epigraphhead[30]{\epigraph{#1}{\textit{#2}}}}


\newcommand{\latex}{\LaTeX}
\newcommand{\tex}{\TeX}
\newcommand{\texbook}{\tex book\  }

\newcommand{\bs}{$backslash$}

\newcommand{\BC}{\textsc{bc}}
\newcommand{\AD}{\textsc{ad}}

\title{CONSTRUCTION \\ MANUAL FOR MEP}
\author{City Center Phase 3a \&3b}
\publisher{AL HABTOOR-Specon}
\date{January 2011} % Delete this line to display the current date


% Generates the index we use before the
% out
\usepackage{makeidx}
\makeindex



%%% ACTIVITY DATES
%% Define a new command for activity key-dates
%% this can be saved for shipout later
\newcommand{\keydate}[2]{#1  #2 \\}
\newcommand{\out}[2]{%
  {\index{key dates!#1}% add to index
\immediate\write\tempfile{\noexpand\keydate{ #1}{#2}}}}




%% We open the file to  to write the key-dates
%% we will use it later to import
\newwrite\tempfile
\immediate\openout\tempfile=keydates.tex

















%experimental

%\usepackage{grid}
% \IfFileExists{txfonts.sty}
%  {\usepackage{txfonts}}
%  {\usepackage{times}}

%\usepackage[fontsize=10pt,baseline=14pt,lines=53]{grid}
%\usepackage{siunitx}

\usepackage{fancyvrb}
\usepackage{float}
\usepackage{wrapfig}

%% Using amsmath and amssymbols for 
%% mathematics
%% they have better macros that those provided in LaTex or TeX
%
\usepackage{amsfonts}

\usepackage{amsmath}
\usepackage{amssymb}[2002/01/22] %% and additional symbols


%% Eventually beginning the document
%% 

\newenvironment{thecode}
{\noindent\rule{10cm}{0.1pt}\verbatim} %before
{\endverbatim\noindent\rule{10cm}{0.1pt}}   %after

\usepackage{longtable}


\begin{comment}


\end{comment}
%%% COMMENTED OUT
\usepackage{inconsolata}  % alternative tt font

%% Set some local commands and colors
\usepackage{colortbl}
\definecolor{green}{rgb}{0.1,0.1,0.1}
%\color{green!40!yellow})

\newcommand{\done}{\cellcolor{teal}done}  %{0.9}
\newcommand{\hcyan}[1]{{\color{teal} #1}}

%% Use the graphics package
% For graphics / images
\usepackage{graphicx}
\setkeys{Gin}{width=\linewidth,totalheight=\textheight,keepaspectratio}
\graphicspath{{graphics/}}


\usepackage[some]{background}
\usepackage{soul}

%% hack not to leave blank pages before chapters
%%


\makeatletter
\def\chapter{\clearpage\thispagestyle{plain}\global\@topnum
  \z@\@afterindentfalse
  \secdef\@chapter\@schapter
}
\makeatother


\definecolor{activity1}{rgb}{0,1.0,0}
\definecolor{activity2}{rgb}{0,0.9,0}
\definecolor{activity3}{rgb}{0,0.85,0}
\definecolor{activity4}{rgb}{0,0.75,0}
\definecolor{activity5}{rgb}{0,0.65,0}


%% Color helper routine
\def\getcolor#1#2{%
\def\zero{0}\def\one{1}\def\two{2}%
\if\zero#1 \gdef\statuscolor{white} \else \gdef\statuscolor{#2}\fi%
\if\two#1  \gdef\statuscolor{red} \fi%
}

\gdef\ascale{0.8}

\gdef\floor#1#2#3#4#5#6{%
\centering%
\parindent0pt%
\tikzpicture[scale=\ascale]%
\def\posx{0.6}%
\getcolor{#2}{activity1}%
\draw node [anchor=south east]{#1}; % places label
\draw[fill=\statuscolor] (0,0) rectangle (0.5,0.5); 
\getcolor{#3}{activity2}
\draw[fill=\statuscolor] (\posx,0.0) rectangle (1.1,0.5);
\getcolor{#4}{activity3}
\draw[fill=\statuscolor] (2*\posx,0.0) rectangle (1.7,0.5);
\getcolor{#5}{activity4}
\draw[fill=\statuscolor] (3*\posx,0.0) rectangle (2.3,0.5);
\getcolor{#6}{activity5}
\draw[fill=\statuscolor] (4*\posx,0.0) rectangle (2.9,0.5);
\endtikzpicture%
}


\begin{document}

\input{./Sections/definitions}

\SetBgContents{Update $13^{th}$ Jan}
\BgThispage
% Front matter
\frontmatter

% r.1 blank page
%\blankpage


% r.3 full title page



% r.5 contents
\maketitle
\tableofcontents
\listoffigures
\listoftables
%\chapter{Document Control}
%\input{./manual/bibliographies}\end{document}
\input{./Sections/document-control}
\input{./Sections/MCD}
\input{./Sections/planning-01}
\input{./Sections/engineering}
\input{./sections/siteworks}
\input{./sections/qaqc}
\input{./sections/communication}
\input{./sections/contractual}
\input{./sections/handover}
\input{./sections/businessedge}
%\input{./manual/blog}
\end{document}
\input{./Sections/update-03}
%Main files start here
\mainmatter %% Very important to include
\input{./Sections/CTC}
\input{./Sections/Materials}
\input{./Sections/introduction}
% \iffalse meta-comment
% Copyright (C) 2011 by YOU <YOU@gmail.com>
% ...
% \fi
% \iffalse
%<*driver>
\ProvidesFile{towers.dtx}
%</driver>
%<package>\NeedsTeXFormat{LaTeX2e}[1999/12/01]
%<package>\ProvidesPackage{towers}
%<*package>
   [2011/09/04 1.01 draws a tower]
%</package>
%<*driver>
\documentclass{ltxdoc}
%\usepackage{towers}
\EnableCrossrefs
\CodelineIndex
\RecordChanges
\begin{document}
 \DocInput{towers.dtx}
 \PrintChanges
 \PrintIndex
\end{document}
%</driver>
% \fi
% \providecommand*{\url}{\texttt}
% \GetFileInfo{towers.dtx}
% \title{The \textsf{Towers} package\thanks{This document
%  corresponds to \textsf{towers}~\fileversion,
%  dated~\filedate.}}
% \author{YOU \url{YOU@gmail.com}}
% \date{\filedate}
% \maketitle
% \begin{abstract}
% The \textsf{towers} package provides commands 
% to draw a high-rise and track construction progress.
% \end{abstract}
% \section{Usage}
% \subsection{Parameters}
%
% \DescribeMacro{\amacro}
% The |doc| package provides a few commands to 
% describe macros.
% \StopEventually{}
%
% \section{Implementation}
%
% \iffalse
%<*package>
% \fi
%
%    \begin{macrocode}
\ProvidesPackage{towers}
\RequirePackage{tikz}
%    \end{macrocode}
% 
% \begin{macro}{\somecommand}
% Here you start writing code
%    \begin{macrocode}
\newcommand{\somecommand}{...}
%    \end{macrocode}
% \end{macro}
% \iffalse
%</package>
% \fi
% \Finale
\endinput

I am not aware of any specific literature related to what you are asking. Here I will provide some pointers as to the normal workflow of preparing a package and then outline some specific issues related to tikz and pgf libraries.

__How to write a package__:

To develop a package with LaTeX is fairly simple. (cls guide)

\NeedsTeXFormat{LaTeX2e}
\ProvidesPackage{tower}[1994/06/01 Standard LaTeX package]

__Literate Programming and packaging your package__

Using `docstrip' and `doc'

If you are going to write a large class or package for LATEX then you should
consider using the doc software which comes with LATEX. 
LATEX classes and
packages written using this can be processed in two ways: they can be run
through LATEX, to produce documentation; and they can be processed with
docstrip, to produce the .cls or .sty file.

The doc software can automatically generate indexes of definitions, indexes
of command use, and change-log lists. It is very useful for maintaining and
documenting large TEX sources. 

__PGF Libraries__

The pgf package provides for what it calls `pgflibraries`. The difference between a library and module is the
following: A library just defines additional objects using the basic layer, whereas a module adds completely
new functionality. For instance, a decoration library defines additional decorations, while a decoration
module defines the whole code for handling decorations.

There is a special command for loading library packages (see page 425)

Example: \string \usepgflibrary{arrows}
This command causes the the pgflibraryhlibrary.code.tex to be loaded for each hlibraryi in the
hlist of librariesi. This means that in order to write your own library, place a file of the appropriate
name somewhere where TEX can find it. LATEX, plain TEX, and ConTEXt users can then use your
library.

You should also consider adding a TikZ library that simply includes your pgf library.

PGF libraries are simply files with the specific extension `filename.code.tex`. They can be generated via the doc/docstrip system or just typed in. They are loaded ...

They require the 
\makeatletter

Good practice to add a definition to show its loaded.

\def\tcblibrary@documentation@loaded{}

In main code you may have to write specific commands

\def\tcb@optionlist{}

\def\tcbuselibrary#1{\tcbset{library/.cd,#1}}

\def\tcb@add@library#1#2{%
  \tcbset{library/#1/.code={\@ifundefined{tcblibrary@#1@loaded}{\input #2}{}}}%
  \DeclareOption{#1}{\edef\tcb@optionlist{\tcb@optionlist,#1}}%
}

\tcb@add@library{listings}{tcblistings.code.tex}
\tcb@add@library{theorems}{tcbtheorems.code.tex}
\tcb@add@library{documentation}{tcbdocumentation.code.tex}

Minimal.
http://tex.stackexchange.com/questions/30168/improving-tikz-user-interface

http://tex.stackexchange.com/questions/30168/improving-tikz-user-interface

%%
%% Copyright (C) 2011 by YOU
%%
%% This file may be distributed and/or modified under the
%% conditions of the LaTeX Project Public License, either
%% version 1.2 of this license or (at your option) any later
%% version. The latest version of this license is in:
%%
%% http://www.latex-project.org/lppl.txt
%%
%% and version 1.2 or later is part of all distributions of
%% LaTeX version 1999/12/01 or later.
%%

\input docstrip.tex
\keepsilent
\usedir{tex/latex/towers}

\preamble

This is a generated file.

Copyright (C) 2011 by YOU

This file may be distributed and/or modified under the
conditions of the LaTeX Project Public License, either
version 1.2 of this license or (at your option) any later
version. The latest version of this license is in:

   http://www.latex-project.org/lppl.txt

and version 1.2 or later is part of all distributions of
LaTeX version 1999/12/01 or later.

\endpreamble

\generate {\file {towers.sty}{\from{towers.dtx}{package}}}

\obeyspaces
\Msg{****************************************************}
\Msg{* *}
\Msg{* Happy TeXing! *}
\Msg{* *}
\Msg{****************************************************}


\endbatchfile
\input{./Sections/towers-rotana}
\input{./Sections/towers-shangrila}
\input{./Sections/towers-merweb}
\input{./Sections/Podium}

\input{./Sections/chilledwater}
\input{./Sections/kahraama}

\input{./Sections/HVAC}

\input{./Sections/carparkventilation}
\input{./Sections/staircasepressurization}
\input{./Sections/kitchenextract}
\input{./Sections/smokeexhaust}
\input{./Sections/generators}
\input{./Sections/drainage}
\input{./Sections/domestic}
\input{./Sections/domestichot}
\input{./Sections/firefighting}


\input{./Sections/electrical}
\input{./Sections/lvsystems}
\input{./Sections/bms} 



\input{./Sections/municipalconnections}
\input{./Sections/civildefence}
\input{./sections/logistics}
\appendix
\input{./Sections/listofservices}
%\input{./Sections/mechanicalengineers}
%\input{./Sections/gantcharts}
%\input{./Sections/charts}
%\input{/.Sections/heatgains}


%% Example
%% 
   
   \out{Close Ceilings Rotana Floor 10}{Line 2}
   \out{Rotana,  power-on}{12$^{th}$ Sept 2010}
   \out{Rotana Qatar Cool switch-on}{12th September}


%% we now read the file and dostuff with it
%% 
\chapter{Summary of  Key Dates}
\immediate\closeout\tempfile
\documentclass{article}
\usepackage{filecontents}
\begin{filecontents}{keydates.dat}

\end{filecontents}
\begin{document}
%% Define a new command for activity key-dates
 %% this can be saved for shipout later
 \newcommand{\keydate}[2]{#1  #2 \\}
 \newcommand{\out}[2]{%
   \immediate\write\tempfile{%
      \noexpand\keydate{ #1}{#2}}}

 %% We open the file to  to write the key-dates
 %% we will use it later to import
\newwrite\tempfile
\immediate\openout\tempfile=keydates.dat


 %% Example
 %% 

\out{Activity 1}{10 Sep 2012}
\out{Activity 2}{12 Sep 2012}
\out{Activity 3}{13 Sep 2012}

 %% finally we close the file
 %%
 \immediate\closeout\tempfile

 %% we now read the file and dostuff with it
 %% 
 \section{Summary of  Key Dates}
 \documentclass{article}
\usepackage{filecontents}
\begin{filecontents}{keydates.dat}

\end{filecontents}
\begin{document}
%% Define a new command for activity key-dates
 %% this can be saved for shipout later
 \newcommand{\keydate}[2]{#1  #2 \\}
 \newcommand{\out}[2]{%
   \immediate\write\tempfile{%
      \noexpand\keydate{ #1}{#2}}}

 %% We open the file to  to write the key-dates
 %% we will use it later to import
\newwrite\tempfile
\immediate\openout\tempfile=keydates.dat


 %% Example
 %% 

\out{Activity 1}{10 Sep 2012}
\out{Activity 2}{12 Sep 2012}
\out{Activity 3}{13 Sep 2012}

 %% finally we close the file
 %%
 \immediate\closeout\tempfile

 %% we now read the file and dostuff with it
 %% 
 \section{Summary of  Key Dates}
 \documentclass{article}
\usepackage{filecontents}
\begin{filecontents}{keydates.dat}

\end{filecontents}
\begin{document}
%% Define a new command for activity key-dates
 %% this can be saved for shipout later
 \newcommand{\keydate}[2]{#1  #2 \\}
 \newcommand{\out}[2]{%
   \immediate\write\tempfile{%
      \noexpand\keydate{ #1}{#2}}}

 %% We open the file to  to write the key-dates
 %% we will use it later to import
\newwrite\tempfile
\immediate\openout\tempfile=keydates.dat


 %% Example
 %% 

\out{Activity 1}{10 Sep 2012}
\out{Activity 2}{12 Sep 2012}
\out{Activity 3}{13 Sep 2012}

 %% finally we close the file
 %%
 \immediate\closeout\tempfile

 %% we now read the file and dostuff with it
 %% 
 \section{Summary of  Key Dates}
 \input{keydates.dat} 


\end{document}  


\end{document}  


\end{document} 



\backmatter
%\bibliography{sample-handout}
%\bibliographystyle{plainnat}
\printindex



%% finally we close the file
%%




\end{document}


If the routines are to be integrated with TeX or a TeX-like system optimization should preferably be done at the paragraph level. Consider the text below given by Bishop in his post. The characteristics of the 'rivers' is an advancing front. If the xy positions of the endings of words is captured and one attempts to draw a line through them. If they intersect is not a river. At its extreme at zero angle one would get a figure such as that given by Holkner and the objective function becomes easier to optimize

http://yallara.cs.rmit.edu.au/~aholkner/paper.pdf























