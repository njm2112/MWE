%%
%% \CharacterTable
%%  {Upper-case    \A\B\C\D\E\F\G\H\I\J\K\L\M\N\O\P\Q\R\S\T\U\V\W\X\Y\Z
%%   Lower-case    \a\b\c\d\e\f\g\h\i\j\k\l\m\n\o\p\q\r\s\t\u\v\w\x\y\z
%%   Digits        \0\1\2\3\4\5\6\7\8\9
%%   Exclamation   \!     Double quote  \"     Hash (number) \#
%%   Dollar        \$     Percent       \%     Ampersand     \&
%%   Acute accent  \'     Left paren    \(     Right paren   \)
%%   Asterisk      \*     Plus          \+     Comma         \,
%%   Minus         \-     Point         \.     Solidus       \/
%%   Colon         \:     Semicolon     \;     Less than     \<
%%   Equals        \=     Greater than  \>     Question mark \?
%%   Commercial at \@     Left bracket  \[     Backslash     \\
%%   Right bracket \]     Circumflex    \^     Underscore    \_
%%   Grave accent  \`     Left brace    \{     Vertical bar  \|
%%   Right brace   \}     Tilde         \~}
%%
% \CheckSum{2048}
% \iffalse
%
% \changes{1.0}{2003/07/02}{First release}
%
% \section{Identification}
%    This is document class Octavo, which can be used with \LaTeX\ only,
%    so prepare to display a message if another \TeX\ format is used.
%    \begin{macrocode}
%<octavo>\NeedsTeXFormat{LaTeX2e}[1995/12/01]
%    \end{macrocode}
%
%    Announce the Class name and its version:
%    \begin{macrocode}
%<octavo>\ProvidesClass{octavo}
%<10pt>\ProvidesFile{oct10.clo}
%<11pt>\ProvidesFile{oct11.clo}
%<12pt>\ProvidesFile{oct12.clo}
%<*driver>
\ProvidesFile{octavo.drv}
%</driver>
              [2007/03/07 v1.2 
%<octavo> a LaTeX document class]
%<10pt|11pt|12pt> a LaTeX file (size option)]
%    \end{macrocode}
%
% \section{The driver for this document}
% Now comes the code for the document driver, that is to say, the file
% which produces the documentation upon extraction from this .dtx file
% by the docstrip program.
%    \begin{macrocode}
%<*driver>
]
\documentclass{ltxdoc}
%    \end{macrocode}
%
%    Not everything has to appear in the index
%    \begin{macrocode}
\DoNotIndex{\',\.,\@M,\@@input,\@Alph,\@alph,\@addtoreset,\@arabic}
\DoNotIndex{\@badmath,\@centercr,\@cite}
\DoNotIndex{\@dotsep,\@empty,\@float,\@gobble,\@gobbletwo,\@ignoretrue}
\DoNotIndex{\@input,\@ixpt,\@m,\@minus,\@mkboth}
\DoNotIndex{\@ne,\@nil,\@nomath,\@plus,\roman,\@set@topoint}
\DoNotIndex{\@tempboxa,\@tempcnta,\@tempdima,\@tempdimb}
\DoNotIndex{\@tempswafalse,\@tempswatrue,\@viipt,\@viiipt,\@vipt}
\DoNotIndex{\@vpt,\@warning,\@xiipt,\@xipt,\@xivpt,\@xpt,\@xviipt}
\DoNotIndex{\@xxpt,\@xxvpt,\\,\ ,\addpenalty,\addtolength,\addvspace}
\DoNotIndex{\advance,\ast,\begin,\begingroup,\bfseries,\bgroup,\box}
\DoNotIndex{\bullet}
\DoNotIndex{\cdot,\cite,\CodelineIndex,\cr,\day,\DeclareOption}
\DoNotIndex{\def,\DisableCrossrefs,\divide,\DocInput,\documentclass}
\DoNotIndex{\DoNotIndex,\egroup,\ifdim,\else,\fi,\em,\endtrivlist}
\DoNotIndex{\EnableCrossrefs,\end,\end@dblfloat,\end@float,\endgroup}
\DoNotIndex{\endlist,\everycr,\everypar,\ExecuteOptions,\expandafter}
\DoNotIndex{\fbox}
\DoNotIndex{\filedate,\filename,\fileversion,\fontsize,\framebox,\gdef}
\DoNotIndex{\global,\halign,\hangindent,\hbox,\hfil,\hfill,\hrule}
\DoNotIndex{\hsize,\hskip,\hspace,\hss,\if@tempswa,\ifcase,\or,\fi,\fi}
\DoNotIndex{\ifhmode,\ifvmode,\ifnum,\iftrue,\ifx,\fi,\fi,\fi,\fi,\fi}
\DoNotIndex{\input}
\DoNotIndex{\jobname,\kern,\leavevmode,\let,\leftmark}
\DoNotIndex{\list,\llap,\long,\m@ne,\m@th,\mark,\markboth,\markright}
\DoNotIndex{\month,\newcommand,\newcounter,\newenvironment}
\DoNotIndex{\NeedsTeXFormat,\newdimen}
\DoNotIndex{\newlength,\newpage,\nobreak,\noindent,\null,\number}
\DoNotIndex{\numberline,\OldMakeindex,\OnlyDescription,\p@}
\DoNotIndex{\pagestyle,\par,\paragraph,\paragraphmark,\parfillskip}
\DoNotIndex{\penalty,\PrintChanges,\PrintIndex,\ProcessOptions}
\DoNotIndex{\protect,\ProvidesClass,\raggedbottom,\raggedright}
\DoNotIndex{\refstepcounter,\relax,\renewcommand}
\DoNotIndex{\rightmargin,\rightmark,\rightskip,\rlap,\rmfamily}
\DoNotIndex{\secdef,\selectfont,\setbox,\setcounter,\setlength}
\DoNotIndex{\settowidth,\sfcode,\skip,\sloppy,\slshape,\space}
\DoNotIndex{\symbol,\the,\trivlist,\typeout,\tw@,\undefined,\uppercase}
\DoNotIndex{\usecounter,\usefont,\usepackage,\vfil,\vfill,\viiipt}
\DoNotIndex{\viipt,\vipt,\vskip,\vspace}
\DoNotIndex{\wd,\xiipt,\year,\z@}
%    \end{macrocode}
%    Generate an index using linenumbers
%    \begin{macrocode}
\EnableCrossrefs
\CodelineIndex
\RecordChanges
%    \end{macrocode}
%    Retrieve date and version information from the file
%    \begin{macrocode}
\GetFileInfo{octavo.drv}
%    \end{macrocode}
%    \begin{macrocode}
\begin{document}
\DocInput{octavo.dtx}
\PrintIndex
\end{document}
%</driver>
%    \end{macrocode}
% \fi
%
% \title{The Octavo Package}
% \author{Stefan A. Revets}
% \date{Version \fileversion, \filedate}
% \maketitle
% \tableofcontents
%
% \section{Introduction}
%
% Donald Knuth developed \TeX\ because of the dissatisfaction he felt
% with the commercial typesetting of his books `The Art of Computer
% programming'~\cite{knuth98}. The program grew into a typesetting system, with 
% a great deal of attention and ingenuity devoted to the calculations of the
% placement of letters, words and paragraphs on a page. More people began
% to use \TeX, and, with the advent of \LaTeX, ready-made class files
% became part of the system. These style files tend to show off the
% capabilities of the system, rather than conform to good design, or, as
% Philip Taylor~\cite{taylor98,taylor99} put it `\ldots all of which shriek 
% ``\TeX'' from every page'.
%
% Class files have been proposed and are available from CTAN which give the more
% fastidious user the means with which to moderate at least some of the excesses 
% of the standard \LaTeX\ styles. The KOMA package is a good example, and so
% is the excellent Memoir class of Peter Wilson~\cite{wilson01}. The manual
% of this package is a worthwhile read, especially since works on design and
% style in book making are not all that easy to obtain.
%
% \subsection{Background}
%
% To \emph{make} a book is an interesting and somewhat involved 
% process~\cite{town}. The text is set in type and printed on paper, the pages
% are gathered and folded into signatures, and these are then bound and covered.
%
% The printing is done on large sheets of paper, which may have to be cut before
% hey are folded into signatures. These are passed on to the the bookbinder who
% sews the signatures into the book proper. As is so often the case with
% venerable crafts, a florid terminology developed which facilitated contacts
% between the guilds of printers and binders, while at the same time inspiring
% some sort of grudging respect in the uninitiated, keeping them at bay.
%
% The sheets of paper, the broadsides, came in a limited number of more or less
% standard sizes. These sheets are known under the names Foolscap, Crown, 
% Post, Demy, Medium, Royal, Super Royal and Imperial. Folding the sheets
% adds another set of terms: a single fold yields a Folio, a double fold a
% Quarto, a triple fold an Octavo and so on. Hence, the combination of a
% sheetname and the number of folds denotes a particular and precise size of
% the book~\cite{town,williamson66}, for example, a crown octavo measures
% 191x127~mm.
%
% Jan Tschichold devoted a lifetime to the design and production of books. 
% Fortunately, he wrote many of his views down. Later in life, he assembled 
% his findings in an overview~\cite{tschichold87}, discussing in a more
% coherent, concentrated fashion the questions which face anyone wishing to
% make a book. Of particular interest is his analysis of size and proportions of
% page, text and margins. Through his own historical investigations, he made, 
% and continues to make, a very strong and convincing case against the more
% `modern' tendency of making squarish books. His elegant, eloquent and well
% illustrated discourse guides us away from the ugliness of the convenient but
% rather square A format (proportion 1:1.414) back to the more rectangular
% formats with  proportions tending towards the Golden  Ratio (from 1:1.5 to
% 1:1.618). It is certainly no accident that the classical sheets of paper
% traditionally used in the printing of books show proportions which fall very
% nicely within this range. 
%
% Tschichold analysed the size and disposition of the text block relative
% to the page as well. By 1953, he published his discovery of the late gothic 
% workshop secrets of determining the margins on a page.
% Rosarivo~\cite{rosarivo61}confirmed this determination through his own, 
% independent discovery of a different method which happily yielded the same 
% results. Rosarivo's method is very easy to use and implement automatically:
% divide the height and width of the page into an equal number, assign one part
% to make up inner and upper margin, and two parts to make up outer and lower
% margin, with the remainder defining the area available for printing the body
% text.
%
% The amount of leading between lines of print depends on the nature of the
% fount, its size, and the length of the line. Also, it should remain fixed for
% the entire text, yielding a constant, regular grid of lines. Tschichold
% discussed the leading of lines of text in some detail, and similar views can 
% be found in Williamson's book~\cite{williamson66}.
%
% It should be clear by now that the standard book class included in \LaTeX\
% requires a fair amount of adjustment to reflect more closely the design
% principles just highlighted. The Octavo class is an attempt to provide
% such a class file.
%
% \subsection{Rationale}
%
% The Octavo class is in essence a modification of the standard \LaTeX\ book
% class, with a similar, limited amount of options and choices. These 
% limitations are there because a number of inbuilt choices have been made on 
% the basis of design principles and guidelines garnered from a variety of
% authors and books. The class is christened Octavo because its express
% aim is to implement many of these classical design elements. 
% 
% First of all, the option of page sizes are the classical octavo sizes. The
% four smallest ones, i.e., foolscap, crown, post and largepost can be made
% to fit two by two on A4 paper, allowing one to produce books in these formats
% on printers routinely available.
%
% The size of the margins follows the analysis by Tschichold, but instead of
% adopting the 6:9 ratio of division of the page (which yields only a 44\%
% covering of a page by text), a 7:10 ratio is here implemented (yielding 
% a 49\% cover of a page by text).
%
% Including a title page is very necessary when making a book, but the way in
% which to do this is not as straightforward a matter as that may seem at first.
% Title pages (plural) are much better when they are treated, and made, as
% an integral part of the endpapers. When binding (good) books, endpapers
% fulfill a functional and aesthetical role, and a separate folio bearing the
% titling, adds to the strength, and beauty, of the endpapers~\cite{town}.
%
% Ideally, a separate folio should be made, bearing on the first recto page the
% bastard title, on the verso possibly a frontispiece, or failing something
% suitable, left blank. The next recto bears title, author, publisher and
% the like, and the final verso provides space for the biblio or imprint. 
% From a design perspective, it is not really feasible, nor desirable, to
% lay down hard and fast rules for the organisation or the placement of the
% various elements needed on a titlepage. In practice, it is much easier 
% to do this as a separate document, especially when the text is to be
% organised into signatures.
%
% An attempt is made here to try and maintain a uniform printing grid across 
% the text. Amongst the \TeX\ algorithms, Knuth developed a set with the aim of
% keeping the overall text block on each page the same size. If necessary, this
% means allowing the change of spacing between paragraphs, around floating
% objects, or indeed the leading within a paragraph. To counteract this 
% behaviour completely would require a rewriting of part of the \TeX\ code,
% which really is too daunting a task. However, it seems possible, through a
% judicious setting of parameters, lengths and sizes, almost to achieve the
% same result. In principle, the octavo class will maintain the regularity
% of the grid within the text body, i.e., the leading should remain constant,
% and successive paragraphs should not become separated by extra space. 
% Adjustments, if and when they become necessary, will be made to the spacing 
% around headings, lists and floats. Nevertheless, there will be times when
% the actual text, the parameters defined by the octavo class, and the \TeX\
% kernel will conspire to produce truly horrible results. Such misfortunes call
% for direct intervention, or, the user willing, suggestions of improving the
% definitions of the parameters proposed here for the octavo class. 
%
% \subsection{Some notes on making books with octavo}
%
% Thanks to excellent works on bookbinding such as the book by Lawrence 
% Town~\cite{town}, the physical making of books lies well within reach of
% anyone caring to make the effort. Thanks to a number of additional programs,
% also available from CTAN, it is a straightforward matter to organise and print 
% out a text in signatures. Here are some of the means which have proved to be
% both useful and effective.
%
% As already mentioned, the formats foolscap, crown, post and largepost can be
% made to sit on an A5 page. This means that four pages can be put on a single
% A4 sheet, a set of which can be made to form a signature. To do so requires 
% a few careful but simple adjustments.
%
% A problem comes up with the imposition of the textpages on an A4 sheet. The
% four formats mentioned are all narrower than an A5 sheet, and hence extra 
% space is left protruding at the outer margins. This is not a problem with the
% odd-numbered pages, as it is their inner margin which is made to coincide with
% the inner margin of the A5 sheet. However, the even-numbered pages have their 
% outer margin made to coincide with the outer margin of the A5 sheet, the 
% latter of course being too wide. What should happen is to have the inner 
% margin of the even-numbered sheet coincide with the inner margins of the A5
% sheet. The simplest solution is to add a single line to the preamble of the
% document: |\addtolength{\evensidemargin}{148.5mm-\paperwidth}|, rather than
% try and reposition only the even-numbered pages with dvidvi, dvips, and the 
% like.
%
% Organising the pages into signatures can be done quite easily. With dvips:
% \begin{enumerate}
%     \item dvips -ta5 \meta{file}
%     \item psbook -s16 \meta{file}.ps \meta{file}.tmp
%     \item psnup -pa4 -Pa5 -l -2 -s1.0 \meta{file}.tmp \meta{file}.ps
% \end{enumerate}
% or with dvihplj or similar, using the switches:\\
% |prthpljh| \meta{file} |+columns:2 +rows:1 +section-size:4|\\
% |+full-last-section +page-width:148.5mm +page-height:210mm|\\ 
% |+landscape-mode +double-sided:odd[even]|
%
% These sets of commands and switches yield signatures made up of 16 pages on 4 
% sheets. Such an arrangement is very effective and handy when binding the
% book. It may be necessary to instruct the printer to shift each page a little
% on the sheet to ensure perfect register, and a little experimenting may be
% called for.
%
% \subsection{Acknowledgements}
% Octavo is a modification of |classes.dtx| written by Leslie Lamport (1992),
% Frank Mittelbach (1994-97) and Johannes Braams (1994-97). As can be seen
% from the code, my own input is restricted to a tweaking of some parameters
% and true credit is due to Lamport, Mittelbach and Braams for their
% monumental efforts.
%
% \begin{thebibliography}{16}
%
% \bibitem{knuth98} Knuth,~D. 1998. \emph{Digital Typography}. CSLI 
% Publications, Stanford.
%
% \bibitem{rosarivo61} Rosarivo,~R. 1961. \emph{Divina proportio typographica}. 
% Scherpe, Krefeld.
%
% \bibitem{taylor98} Taylor,~P. 1998. \emph{Book design for \TeX\ users, Part 1: 
% Theory.} TUGBoat, 19:65--74.
%
% \bibitem{taylor99} Taylor,~P. 1999. \emph{Book design for \TeX\ users, Part 2:
% Practice.} TUGBoat, 20:378--389.
%
% \bibitem{town} Town,~L. \emph{Bookbinding by hand.} Faber \& Faber, London.
%
% \bibitem{tschichold87} Tschichold,~ J. 1987. \emph{Ausgew\"{a}hlte Aufs\"{a}tze
% \"{u}ber Fragen der Gestalt des Buches und der Typographie}. Birkh\"{a}user
% Verlag, Basel.
%
% \bibitem{williamson66} Williamson,~H. 1966. \emph{Methods of book design.} Oxford 
% University Press, Oxford.
%
% \bibitem{wilson01} Wilson,~P. 2001. \emph{The Memoir class for configurable
% typesetting.} CTAN. |macros\latex\contrib\memoir|
%
% \end{thebibliography}
% \StopEventually{}
%
%
% \section{Preliminaries}
%    We begin by declaring a number of commands, parameters and control 
%    sequences which will be used later on. 
%
%    We load the `xkeyval' package. This simplifies the user interface for the
%    document class options
%    The `xkeyval' package simplifies the user interface for the document class options
\RequirePackage{xkeyval}






% \begin{macro}{\@ptsize}
%    The controlsequence stores the second digit of the pointsize in which
%    the body of the text will be typeset.
%    \begin{macrocode}
%<*octavo>
\newcommand\@ptsize{}
%    \end{macrocode}
% \end{macro}
%
% \begin{macro}{\if@restonecol}
%    In texts printed in two columns, this switch is used to remember to
%    switch back to two column if there has been a temporary change to single
%    column mode.
%    \begin{macrocode}
\newif\if@restonecol
%    \end{macrocode}
% \end{macro}
%
% \begin{macro}{\if@titlepage}
%    Usually, a titlepage is produced with a book. However, if a set of
%    endpapers is made, titlepages can be made separately from the body 
%    of the book, and it is useful to omit the production of a titlepage.
%    \begin{macrocode}
\newif\if@titlepage
\@titlepagetrue
%    \end{macrocode}
% \end{macro}
%
% \begin{macro}{\if@openright}
% \begin{macro}{\if@openbib}
%    Another set of switches, to indicate if new chapters must start on a
%    right-hand page. Normally they do, but novels and the like may well
%    benefit from having new chapters start on either side.
%    \begin{macrocode}
\newif\if@openright
\newif\if@openbib
\@openbibfalse
%    \end{macrocode}
% \end{macro}
% \end{macro}
%
% \begin{macro}{\if@mainmatter}
%    A switch to indicate when we are processing the main material of the book.
%    \begin{macrocode}
\newif\if@mainmatter
\@mainmattertrue
%    \end{macrocode}
% \end{macro}
%
% \section{Class options}
% \subsection{Declaring options}
%    The options available to the octavo class are essentially the same as those
%    defined for the standard book class. What is different and new is a set of
%    papersize options.
%
% \begin{macro}{\paperheight}
% \begin{macro}{\paperwidth}
%    We begin with the papersizes. These are largely the traditional octavo
%    sizes. The option |largepost| is slightly different: it is an example
%    of Tschichold's pentagon-based proportion (1:1.538) applied to a page
%    height of an a5 page. 
%    \begin{macrocode}
\DeclareOption{foolscap}
    {\setlength\paperheight{171mm}%
    \setlength\paperwidth{108mm}}
\DeclareOption{crown}
    {\setlength\paperheight{191mm}%
    \setlength\paperwidth{127mm}}
\DeclareOption{post}
    {\setlength\paperheight{194mm}%
    \setlength\paperwidth{122mm}}
\DeclareOption{largepost}
    {\setlength\paperheight{210mm}%
    \setlength\paperwidth{137mm}}
\DeclareOption{demy}
    {\setlength\paperheight{222mm}%
    \setlength\paperwidth{143mm}}
\DeclareOption{medium}
    {\setlength\paperheight{229mm}%
    \setlength\paperwidth{146mm}}
\DeclareOption{royal}
    {\setlength\paperheight{254mm}%
    \setlength\paperwidth{159mm}}
\DeclareOption{superroyal}
    {\setlength\paperheight{267mm}%
    \setlength\paperwidth{171mm}}
\DeclareOption{imperial}
    {\setlength\paperheight{279mm}%
    \setlength\paperwidth{191mm}}
%    \end{macrocode}
% \end{macro}
% \end{macro}
%
% \begin{macro}{\@ptsize}
%    The choice of type size is kept through the last digit of the size
%    in |\@ptsize|. There are apparently good, or at least, historical
%    reasons for doing so. Since this may well ensure compatibility with
%    other packages, the method is maintained here.
%    \begin{macrocode}
\DeclareOption{10pt}{\renewcommand\@ptsize{0}}
\DeclareOption{11pt}{\renewcommand\@ptsize{1}}
\DeclareOption{12pt}{\renewcommand\@ptsize{2}}
%    \end{macrocode}
% \end{macro}
%
% \begin{macro}{twoside}
%    When printing books, two-sided printing is the norm. This
%    means that we have to set the |if@mparswitch| to get any margin
%    paragraphs into the outside margin.
%    \begin{macrocode}
\DeclareOption{twoside}{\@twosidetrue \@mparswitchtrue}
%    \end{macrocode}
% \end{macro}
%
% \begin{macro}{draft}
%    When preparing a text for final tweaking, the |draft| mode may come in
%    handy, as it will show on the page where overfull boxes appear.
%    \begin{macrocode}
\DeclareOption{draft}{\setlength\overfullrule{5pt}}
\DeclareOption{final}{\setlength\overfullrule{0pt}}
%    \end{macrocode}
% \end{macro}
%
% \begin{macro}{titlepage}
%    Having argued that it is much better to deal with the titlepage separately
%    from the book itself, we do leave the choice of a titlepage as integral 
%    part of the book as an option. However, the default will be to have no
%    titlepage.
%    \begin{macrocode}
\DeclareOption{titlepage}{\@titlepagetrue}
\DeclareOption{notitlepage}{\@titlepagefalse}
%    \end{macrocode}
% \end{macro}
% 
% \begin{macro}{openright}
%    New chapters normally start on a right-hand page in books, but novels, or
%    poetry books may well be better off by allowing chapters to begin on either
%    side.
%    \begin{macrocode}
\DeclareOption{openright}{\@openrighttrue}
\DeclareOption{openany}{\@openrightfalse}
%    \end{macrocode}
% \end{macro}
%
% \begin{macro}{twocolumn}
%    Single or double column arrangement of the body text is chosen with this
%    option. The choice does require some care, as the width of a line of text
%    that can be read comfortably depends on the total number of ens, and this
%    is a function of the fount, the fount size and the text width. The larger
%    formats almost always benefit from being typeset in two columns.
%    \begin{macrocode}
\DeclareOption{onecolumn}{\@twocolumnfalse}
\DeclareOption{twocolumn}{\@twocolumntrue}
%    \end{macrocode}
% \end{macro}
%
% \begin{macro}{leqno}
% \begin{macro}{fleqn}
%    When including mathematical formul\ae , equation numbers are usually
%    placed to the right, but the |leqno| option places them on the left of the
%    the equation. The |fleqn| option redefines the math display environment so
%    that equations are typeset flush with the left margin and indented by
%    |\mathindent| from the prevailing left margin.
%    \begin{macrocode}
\DeclareOption{leqno}{\input{leqno.clo}}
\DeclareOption{fleqn}{\input{fleqn.clo}}
%    \end{macrocode}
% \end{macro}
% \end{macro}
%
% \begin{macro}{openbib}
%    The format of the bibliography is set to an `open' one, that is, each block
%    starts on a new line, with successive lines in a block indented by the
%    amount defined by |\bibindent|.
%    \begin{macrocode}
\DeclareOption{openbib}{\@openbibtrue}
%    \end{macrocode}
% \end{macro}
%
% \subsection{Executing options}
%    Here, default options are executed in order to initialise a number of
%    variables. The |\ProcessOptions| command then ensures the execution of the
%    code for every option declared and for which the user included an option
%    in the |\documentclass| command.
%    \begin{macrocode}
\ExecuteOptions{crown,10pt,twoside,final,notitlepage,openright,onecolumn}
\ProcessOptions
%    \end{macrocode}
%    After the execution of all the options, we can now load the class option
%    file which contains all the size dependent code.
%    \begin{macrocode}
\input{oct1\@ptsize.clo}
%</octavo>
%    \end{macrocode}
%
% \section{Document layout}
% \subsection{Fount sizes}
%    The user has access to a number of commands which change the size of
%    the fount, relative to the `main' size used for the bulk of the text.
%    These |\size| commands issue a 
%    |\@setfontsize||\size|\meta{font-size}\meta{baselineskip} where:
%    \begin{description}
%    \item{\meta{font-size}} The absolute size of the fount to use from
%        now on.
%    \item{\meta{baselineskip}} The normal value of |\baselineskip|
%        for the size of the fount selected. (The actual value will be
%        |\baselinestretch| * \meta{baselineskip}.)
%    \end{description}
%
%    A number of commands, defined in the \LaTeX\ kernel, shorten the
%    following  definitions and are used throughout. These are:
%    \begin{center}
%    \begin{tabular}{ll@{\qquad}ll@{\qquad}ll}
%    \verb=\@vpt= & 5 & \verb=\@vipt= & 6 & \verb=\@viipt= & 7 \\
%    \verb=\@viiipt= & 8 & \verb=\@ixpt= & 9 & \verb=\@xpt= & 10 \\
%    \verb=\@xipt= & 10.95 & \verb=\@xiipt= & 12 & \verb=\@xivpt= & 14.4\\
%    \ldots
%    \end{tabular}
%    \end{center}
%
% \begin{macro}{\normalsize}
% \begin{macro}{\@normalsize}
%    The user command to obtain the `main' size is |\normalsize|. \LaTeX\
%    uses |\@normalsize| when referring to the main size and maintains this
%    value even if |\normalsize| is redefined. The |\normalsize| macro also
%    sets values for |\abovedisplayskip|, |\abovedisplayshortskip| and 
%    |\belowdisplayshortskip|.
%    \begin{macrocode}
%<*10pt|11pt|12pt>
\renewcommand\normalsize{%
%<*10pt>
    \@setfontsize\normalsize\@xpt\@xiipt
    \abovedisplayskip 12\p@ \@plus 2\p@ \@minus 2\p@
    \abovedisplayshortskip \z@ \@plus 3\p@
    \belowdisplayshortskip 6\p@ \@plus 3\p@ \@minus 3\p@
%</10pt>
%<*11pt>
    \@setfontsize\normalsize\@xipt{13.2}%
    \abovedisplayskip 13.2\p@ \@plus 3\p@ \@minus 3\p@
    \abovedisplayshortskip \z@ \@plus 3\p@
    \belowdisplayshortskip 6.6\p@ \@plus 3\p@ \@minus 3\p@
%</11pt>
%<*12pt>
    \@setfontsize\normalsize\@xiipt\@xivpt
    \abovedisplayskip 14.4\p@ \@plus 3\p@ \@minus 3\p@
    \abovedisplayshortskip \z@ \@plus 3\p@
    \belowdisplayshortskip 7.2\p@ \@plus 3\p@ \@minus 3\p@
%</12pt>
%    \end{macrocode}
%    The |\belowdisplayskip| equals |\abovedisplayskip|. The parameters of
%    the first-level list are always given by |\@listI|.
%    \begin{macrocode}
    \belowdisplayskip \abovedisplayskip
    \let\@listi\@listI}
%    \end{macrocode}
%    Initially, the |normalsize| fount is put into use for the document
%    \begin{macrocode}
\normalsize
%    \end{macrocode}
% \end{macro}
% \end{macro}
%
% \begin{macro}{\small}
%    Similar to |\normalsize|
%    \begin{macrocode}
\newcommand\small{%
%<*10pt>
    \@setfontsize\small\@ixpt\@xipt%
    \abovedisplayskip 11\p@ \@plus 2\p@ \@minus2\p@
    \abovedisplayshortskip \z@ \@plus 3\p@
    \belowdisplayshortskip 5.5\p@ \@plus 2\p@ \@minus 2\p@
    \def\@listi{\leftmargin\leftmargini
        \topsep 5.5\p@ \@plus 2\p@ \@minus 2\p@
        \parsep  \z@
        \itemsep \z@}%
%</10pt>
%<*11pt>
    \@setfontsize\small\@xpt\@xiipt
    \abovedisplayskip 12\p@ \@plus 2\p@ \@minus 2\p@
    \abovedisplayshortskip \z@ \@plus 3\p@
    \belowdisplayshortskip 6\p@ \@plus 3\p@ \@minus 3\p@
    \def\@listi{\leftmargin\leftmargini
        \topsep 6\p@ \@plus 2\p@ \@minus 2\p@
        \parsep  \z@
        \itemsep \z@}%
%</11pt>
%<*12pt>
    \@setfontsize\small\@xipt{13.2}%
    \abovedisplayskip 13.2\p@ \@plus 3\p@ \@minus 3\p@
    \abovedisplayshortskip \z@ \@plus 3\p@
    \belowdisplayshortskip 6.6\p@ \@plus 3\p@ \@minus 3\p@
    \def\@listi{\leftmargin\leftmargini
        \topsep 6.6\p@ \@plus 3\p@ \@minus 3\p@
        \parsep  \z@
        \itemsep \z@}%
%</12pt>
    \belowdisplayskip \abovedisplayskip}
%    \end{macrocode}
% \end{macro}
%
% \begin{macro}{\footnotesize}
%    Again, very similar to |\normalsize|
%    \begin{macrocode}
\newcommand\footnotesize{%
%<*10pt>
    \@setfontsize\footnotesize\@viiipt{9.5}%
    \abovedisplayskip 9.5\p@ \@plus 2\p@ \@minus 2\p@
    \abovedisplayshortskip \z@ \@plus 2\p@
    \belowdisplayshortskip 5\p@ \@plus 2\p@ \@minus 2\p@
    \def\@listi{\leftmargin\leftmargini
        \topsep  5\p@ \@plus 2\p@ \@minus 2\p@
        \parsep  \z@
        \itemsep \z@}%
%</10pt>
%<*11pt>
    \@setfontsize\footnotesize\@ixpt\@xipt%
    \abovedisplayskip 11\p@ \@plus 2\p@ \@minus 2\p@
    \abovedisplayshortskip \z@ \@plus 2\p@
    \belowdisplayshortskip 5.5\p@ \@plus 2\p@ \@minus 2\p@
    \def\@listi{\leftmargin\leftmargini
        \topsep 5.5\p@ \@plus 2\p@ \@minus 2\p@
        \parsep  \z@
        \itemsep \z@}%
%</11pt>
%<*12pt>
    \@setfontsize\footnotesize\@xpt\@xiipt
    \abovedisplayskip 12\p@ \@plus 2\p@ \@minus 2\p@
    \abovedisplayshortskip \z@ \@plus 3\p@
    \belowdisplayshortskip 6\p@ \@plus 3\p@ \@minus 3\p@
    \def\@listi{\leftmargin\leftmargini
        \topsep 6\p@ \@plus 2\p@ \@minus 2\p@
        \parsep  \z@
        \itemsep \z@}%
%</12pt>
    \belowdisplayskip \abovedisplayskip}
%    \end{macrocode}
% \end{macro}
%
% \begin{macro}{\scriptsize}
% \begin{macro}{\tiny}
% \begin{macro}{\large}
% \begin{macro}{\Large}
% \begin{macro}{\LARGE}
% \begin{macro}{\huge}
% \begin{macro}{\Huge}
%    These macros select new fountsizes, but leave display and lists alone.
%    \begin{macrocode}
%<*10pt>
\newcommand\scriptsize{\@setfontsize\scriptsize\@viipt\@viiipt}
\newcommand\tiny{\@setfontsize\tiny\@vpt\@vipt}
\newcommand\large{\@setfontsize\large\@xiipt{14}}
\newcommand\Large{\@setfontsize\Large\@xivpt{18}}
\newcommand\LARGE{\@setfontsize\LARGE\@xviipt{22}}
\newcommand\huge{\@setfontsize\huge\@xxpt{25}}
\newcommand\Huge{\@setfontsize\Huge\@xxvpt{30}}
%</10pt>
%<*11pt>
\newcommand\scriptsize{\@setfontsize\scriptsize\@viiipt{9.5}}
\newcommand\tiny{\@setfontsize\tiny\@vipt\@viipt}
\newcommand\large{\@setfontsize\large\@xiipt{14}}
\newcommand\Large{\@setfontsize\Large\@xivpt{18}}
\newcommand\LARGE{\@setfontsize\LARGE\@xviipt{22}}
\newcommand\huge{\@setfontsize\huge\@xxpt{25}}
\newcommand\Huge{\@setfontsize\Huge\@xxvpt{30}}
%</11pt>
%<*12pt>
\newcommand\scriptsize{\@setfontsize\scriptsize\@viiipt{9.5}}
\newcommand\tiny{\@setfontsize\tiny\@vipt\@viipt}
\newcommand\large{\@setfontsize\large\@xivpt{18}}
\newcommand\Large{\@setfontsize\Large\@xviipt{22}}
\newcommand\LARGE{\@setfontsize\LARGE\@xxpt{25}}
\newcommand\huge{\@setfontsize\huge\@xxvpt{30}}
\let\Huge=\huge
%</12pt>
%</10pt|11pt|12pt>
%    \end{macrocode}
% \end{macro}
% \end{macro}
% \end{macro}
% \end{macro}
% \end{macro}
% \end{macro}
% \end{macro}
%
% \subsubsection{Fount changes}
% Fount changes should really be done with the \LaTeXe\ commands, but quite
% a few packages rely on the older versions of these commands. Therefore, for
% the sake of compatibility, the old commands are retained.
% \begin{macro}{\rm}
% \begin{macro}{\tt}
% \begin{macro}{\sf}
%    These are commands to change the fount family.
%    \begin{macrocode}
%<*octavo>
\DeclareOldFontCommand{\rm}{\normalfont\rmfamily}{\mathrm}
\DeclareOldFontCommand{\sf}{\normalfont\sffamily}{\mathsf}
\DeclareOldFontCommand{\tt}{\normalfont\ttfamily}{\mathtt}
%    \end{macrocode}
% \end{macro}
% \end{macro}
% \end{macro}
%
% \begin{macro}{\bf}
% The command to change to the bold series
%    \begin{macrocode}
\DeclareOldFontCommand{\bf}{\normalfont\bfseries}{\mathbf}
%    \end{macrocode}
% \end{macro}
%
% \begin{macro}{\sl}
% \begin{macro}{\it}
% \begin{macro}{\sc}
% And finally, the commands to change the shape of the founts.
%    \begin{macrocode}
\DeclareOldFontCommand{\sl}{\normalfont\slshape}{\@nomath\sl}
\DeclareOldFontCommand{\it}{\normalfont\itshape}{\mathit}
\DeclareOldFontCommand{\sc}{\normalfont\scshape}{\@nomath\sc}
%    \end{macrocode}
% \end{macro}
% \end{macro}
% \end{macro}
%
% \subsection{Paragraphs}
%    The parameters which control \TeX 's behaviour when typesetting
%    paragraphs receive a bit of a tweak here. Contrary to the usual
%    behaviour of modifying the grid with glue when difficulties are
%    encountered with vertical space, here we shall try to counteract
%    these tendencies and enforce as much as possible uniformity of the 
%    grid of lines.
% \begin{macro}{\lineskip}
% \begin{macro}{\normallineskip}
%    These two parameters influence \TeX when two lines come too close.
%    \begin{macrocode}
\setlength\lineskip{1\p@}
\setlength\normallineskip{1\p@}
%    \end{macrocode}
% \end{macro}
% \end{macro}
%
% \begin{macro}{\baselinestretch}
%    Used by \TeX\ as a multiplier for |\baselineskip|. \emph{Default}
%    behaviour is not to stretch the baselines. If this command does not
%    resolve to `empty', any |plus| or |minus| part in the specification
%    of |\baselineskip| is ignored.
%    \begin{macrocode}
\renewcommand\baselinestretch{}
%</octavo>
%    \end{macrocode}
% \end{macro}
%
% \begin{macro}{\parskip}
%    Extra vertical space can be put between successive paragraphs with
%    |\parskip| as it is usually defined as a zero rubber length. Here, it 
%    is set to zero, without rubber extention, hopefully helping to maintain
%    the regular grid. 
%    \begin{macrocode}
%<*10pt|11pt|12pt>
\setlength\parskip{0\p@}
%    \end{macrocode}
% \end{macro}
%
% \begin{macro}{\parindent}
%    The value of |\parindent| was set to a variety of sizes depending on the
%    fount size and whether single- or double-column is in use. Using the
%    |em| measure relates the indent to the fount in use, and there is little
%    merit in having a marginally wider indent when typesetting in a single 
%    column.
%    \begin{macrocode}
\setlength\parindent{1em}
%</10pt|11pt|12pt>
%    \end{macrocode}
% \end{macro}
%
% \begin{macro}{\@lowpenalty}
% \begin{macro}{\@medpenalty}
% \begin{macro}{\@highpenalty}
%    Command such as |\nopagebreak| or |\nolinebreak| put in penalties to
%    discourage the placement of these breaks. Depending on the argument
%    with which they are put, a low, medium or high penalty is chosen.
%    \begin{macrocode}
%<*octavo>
\@lowpenalty 51
\@medpenalty 151
\@highpenalty 301
%</octavo>
%    \end{macrocode}
% \end{macro}
% \end{macro}
% \end{macro}
%
%
% \subsection{Page layout}
%    Here the class file deviates more noticably from the standard book class.
%    To begin with, all dimensions are measured from the true upper left corner 
%    of the page. Dimensions and placement of the text area is directly related
%    to and determined as proportions of the dimensions of the page.
%
% \changes{1.2a}{2007/03/07}{Removed changes to hoffset and voffset to maintain 
%               compatibility with other class files, notably crop}
%
% \subsubsection{Vertical spacing}
%
% \begin{macro}{\headheight}
% \begin{macro}{\headsep}
% \begin{macro}{\topskip}
%    The |\headheight| is the height of the box containing the running head.
%    The |\headsep| is the distance between the bottom of the box with the
%    running head and the top of the text. The |\topskip| is the
%    |\baselineskip| for the first line on the page.
%    \begin{macrocode}
\setlength\headheight{12\p@}
%<10pt>\setlength\headsep{6\p@}
%<11pt>\setlength\headsep{7\p@}
%<12pt>\setlength\headsep{8\p@}
%<10pt>\setlength\topskip{10\p@}
%<11pt>\setlength\topskip{11\p@}
%<12pt>\setlength\topskip{12\p@}
%    \end{macrocode}
% \end{macro}
% \end{macro}
% \end{macro}
%
% \begin{macro}{\footskip}
%    This is the distance from the baseline of the box containing the running
%    foot to the bottom of the text.
%    \begin{macrocode}
%<10pt>\setlength\footskip{25\p@}
%<11pt>\setlength\footskip{27.5\p@}
%<12pt>\setlength\footskip{30\p@}
%    \end{macrocode}
% \end{macro}
%
% \begin{macro}{\maxdepth}
%    The |\maxdepth| register has a function similar to that of |\topskip|. The
%    register |\@maxdepth| should always contain a copy of |\maxdepth|, and this
%    is ensured by setting it when issuing the command |\begin{document}|. In
%    \TeX\ and \LaTeX2.09 it was fixed at |4pt|. Following \LaTeXe, it now
%    depends on the typesize, and calculation simplifies it to half the topskip.
%    \begin{macrocode}
\setlength\maxdepth{0.5\topskip}
%    \end{macrocode}
% \end{macro}
%
% \subsubsection{Text dimensions}
%    Here we implement the proposals and calculations by Tschichold. In
%    particular, the proportions of text to page are set so that 49\% of
%    the page area is covered by the text block.
%
% \begin{macro}{\topmargin}
%    The |\topmargin| is the distance between the top of the text block and
%    the top of the box containing the running head. The running head is not
%    part of the text block, and it goes into the top margin. As |\topmargin| 
%    is measured from the top of the box, we have to correct for header height
%    and header separation.
%    By subtracting 1 inch, the effect of voffset is overcome.
% \changes{1.2b}{2007/03/07}{Subtract 1 inch from |topmargin| to move textblock}
%    \begin{macrocode}
\setlength{\topmargin}{0.1\paperheight}
    \addtolength{\topmargin}{-\headheight}
    \addtolength{\topmargin}{-\headsep}
    \addtolength{\topmargin}{-1in}
%    \end{macrocode}
% \end{macro}
%
% \begin{macro}{\textwidth}
%    Define the width of the text block to 0.7 of the page width, and make
%    calculations a little easier by adjusting the calculated width to a 
%    whole number of points.
%    \begin{macrocode}
\setlength{\textwidth}{0.7\paperwidth}
    \@settopoint\textwidth
%    \end{macrocode}
% \end{macro}
%
% \begin{macro}{\textheight}
%    The height of the text block itself is set to 0.7 times the page height. 
%    This amount is then adjusted to ensure that a whole number of lines makes 
%    up the text block, and does so exactly.
%    \begin{macrocode}
\setlength\@tempdima{0.7\paperheight}
%    \end{macrocode}
%    take away the first line, which is a bit shorter than the |\baselineskip|,
%    \begin{macrocode}
    \addtolength\@tempdima{-\topskip}
%    \end{macrocode}
%    this length may be very close, but just a little too small to accommodate 
%    one more line, so we add a small amount,
%    \begin{macrocode}
    \addtolength\@tempdima{5\p@}
%    \end{macrocode}
%    and calculate the number of lines in this length,
%    \begin{macrocode}
    \divide\@tempdima\baselineskip
    \@tempcnta=\@tempdima
%    \end{macrocode}
%    The correct textheight comes to the number of lines just calculated, 
%    multiplied by the height of text lines, |\baselineskip|, and with the 
%    addition of the |\topskip| we took away initially.
%    \begin{macrocode}
    \setlength\textheight{\@tempcnta\baselineskip}
    \addtolength\textheight{\topskip}
%    \end{macrocode}
% \end{macro}
%
% \subsubsection{Margin dimensions}
%     Now that we have set the size of the text block, the amount of space
%     available for margins is set as well. The remaining white space is divided
%     in a 1:2 ratio, hence the proportions between margins and text become 1:7:2.
%
% \begin{macro}{\evensidemargin}
% \begin{macro}{\oddsidemargin}
%    Since we are typesetting books, both even and odd side margins have to be
%    set.
% \changes{1.2c}{2007/03/07}{Subtract 1 inch from |oddsidemargin| and |evensidemargin|
%          to move textblock}
%    \begin{macrocode}
\setlength{\evensidemargin}{0.2\paperwidth}
\addtolength{\evensidemargin}{-1in}
\setlength{\oddsidemargin}{0.1\paperwidth}
\addtolength{\oddsidemargin}{-1in}
%    \end{macrocode}
% \end{macro}
% \end{macro}
%
% \begin{macro}{\marginparsep}
% \begin{macro}{\marginparpush}
%    The horizontal space between the text block and marginal notes is set by
%    |\marginparsep|, while the minimal vertical separation between successive
%    marginal notes is controlled by |\marginparpush|.
%    \begin{macrocode}
\if@twocolumn
    \setlength\marginparsep{10\p@}
\else
%<10pt>\setlength\marginparsep{10\p@}
%<11pt>\setlength\marginparsep{11\p@}
%<12pt>\setlength\marginparsep{12\p@}
\fi
%<10pt|11pt>\setlength\marginparpush{5\p@}
%<12pt>\setlength\marginparpush{7\p@}
%    \end{macrocode}
% \end{macro}
% \end{macro}
%
% \begin{macro}{\marginparwidth}
%    The amount of space available for marginal notes is the outer margin.
%    From this we have to substract the separation between notes and the text
%    body, as well as another, marginal outer margin. In principle, marginal
%    text could run up all the way to the outer margin, but this is both ugly
%    and impractical for the book binder. Since the margins have to be cut, it
%    is very wise to leave sufficient space: |30pt| seems a good minimum.
%    \begin{macrocode}
\setlength\@tempdima{\paperwidth}
    \addtolength\@tempdima{-\textwidth}
    \addtolength\@tempdima{-\oddsidemargin}
    \addtolength\@tempdima{-\marginparsep}
    \addtolength\@tempdima{-30\p@}
    \setlength\marginparwidth{\@tempdima}
    \@settopoint\marginparwidth
%    \end{macrocode}
% \end{macro}
%
% \subsubsection{Footnotes}
%
% \begin{macro}{\footnotesep}
%    The height of the strut placed at the beginning of every footnote. For
%    each size, |\footnotesep| is set to the height of the corresponding
%    |\footnotesize| strut, so there is no extra space between successive
%    footnotes.
%    \begin{macrocode}
%<10pt>\setlength\footnotesep{6.65\p@}
%<11pt>\setlength\footnotesep{7.7\p@}
%<12pt>\setlength\footnotesep{8.4\p@}
%    \end{macrocode}
% \end{macro}
%
% \begin{macro}{\footins}
%    The space between the last line of the main text and the top of the first
%    footnote.
%    \begin{macrocode}
%<10pt>\setlength{\skip\footins}{9\p@ \@plus 4\p@ \@minus 2\p@}
%<11pt>\setlength{\skip\footins}{10\p@ \@plus 4\p@ \@minus 2\p@}
%<12pt>\setlength{\skip\footins}{11\p@ \@plus 4\p@ \@minus 2\p@}
%    \end{macrocode}
% \end{macro}
%
% \subsubsection{Float placement}
%    Float placement parameters are given values by \LaTeXe, therefore any
%    parameters which are not counters have to be (re)set with |\renewcommand|.
% \begin{macro}{topnumber}
%    The maximum number of floats allowed at the top of a page.
%    \begin{macrocode}
%<*octavo>
\setcounter{topnumber}{2}
%    \end{macrocode}
% \end{macro}
%
% \begin{macro}{\topfraction}
%    The maximum part of a text page that may be occupied by floats at the
%    top of a page.
%    \begin{macrocode}
\renewcommand\topfraction{.7}
%    \end{macrocode}
% \end{macro}
%
% \begin{macro}{bottomnumber}
%    The equivalent of |topnumber|, but for the maximum number of floats
%    allowed at the bottom of a page.
%    \begin{macrocode}
\setcounter{bottomnumber}{1}
%    \end{macrocode}
% \end{macro}
%
% \begin{macro}{\bottomfraction}
%    The maximum part of a text page that may be occupied by floats at the
%    bottom of a page.
%    \begin{macrocode}
\renewcommand\bottomfraction{.5}
%    \end{macrocode}
% \end{macro}
%
% \begin{macro}{totalnumber}
%    The maximum number of floats in total which may appear on a page.
%    \begin{macrocode}
\setcounter{totalnumber}{3}
%    \end{macrocode}
% \end{macro}
%
% \begin{macro}{\textfraction}
%    The minimum part of a text page which has to be occupied by text.
%    \begin{macrocode}
\renewcommand\textfraction{.2}
%    \end{macrocode}
% \end{macro}
%
% \begin{macro}{\floatpagefraction}
%    The minimum part of a page which has to be occupied by floating object
%    before a `float page' is produced.
%    \begin{macrocode}
\renewcommand\floatpagefraction{.5}
%    \end{macrocode}
% \end{macro}
%
% \begin{macro}{dbltopnumber}
%    The maximum number of two-column floats which may appear at the top of
%    a two-column text page.
%    \begin{macrocode}
\setcounter{dbltopnumber}{2}
%    \end{macrocode}
% \end{macro}
%
% \begin{macro}{\dbltopfraction}
%    The maximum part of a two-column text page which may be occupied at the
%    top by two-column floats.
%    \begin{macrocode}
\renewcommand\dbltopfraction{.7}
%    \end{macrocode}
% \end{macro}
%
% \begin{macro}{\dblfloatpagefraction}
%    The minimum part of a page which has to be occupied by two-column floating
%    objects before a `float page' is produced.
%    \begin{macrocode}
\renewcommand\dblfloatpagefraction{.5}
%</octavo>
%    \end{macrocode}
% \end{macro}
%
% \begin{macro}{\floatsep}
% \begin{macro}{\textfloatsep}
% \begin{macro}{\intextsep}
%    These parameters control the separation between a float object and the
%    other objects on the same page. They are used in both single-column and
%    double-column mode.
%    \begin{description}
%    \item{|\floatsep|} is the amount of space between adjacent floats moved to 
%        the top or to the bottom of the text page.
%    \item{|\textfloatsep|} is the amount of space between the main text and the
%        floats at the top or at the bottom of the text page.
%    \item{|\intextsep|} is the amount of space between in-text floats and the 
%        text
%    \end{description}
%    \begin{macrocode}
%<*10pt>
\setlength\floatsep{12\p@ \@plus 2\p@ \@minus 2\p@}
\setlength\textfloatsep{24\p@ \@plus 2\p@ \@minus 4\p@}
\setlength\intextsep{12\p@ \@plus 2\p@ \@minus 2\p@}
%</10pt>
%<*11pt>
\setlength\floatsep{13.2\p@ \@plus 2\p@ \@minus 2\p@}
\setlength\textfloatsep{26\p@ \@plus 2\p@ \@minus 4\p@}
\setlength\intextsep{13.2\p@ \@plus 2\p@ \@minus 2\p@}
%</11pt>
%<*12pt>
\setlength\floatsep{14.4\p@ \@plus 2\p@ \@minus 2\p@}
\setlength\textfloatsep{28\p@ \@plus 2\p@ \@minus 4\p@}
\setlength\intextsep{14.4\p@ \@plus 4\p@ \@minus 2\p@}
%</12pt>
%    \end{macrocode}
% \end{macro}
% \end{macro}
% \end{macro}
%
% \begin{macro}{\dblfloatsep}
% \begin{macro}{\dbltextfloatsep}
%    In two-column mode, floats spanning the entire |\textwidth| are placed
%    with the parameters |\dblfloatsep| and |\dbltextfloatsep| in control.
%    \begin{description}
%    \item{|\dblfloatsep|} is the amount of space between adjacent floats at the 
%        top or at the bottom of the text page
%    \item{|\dbltextfloatsep|} is the amount of space between the main text and 
%        floats at the top or at the bottom of the text page.
%    \end{description}
%    \begin{macrocode}
%<*10pt>
\setlength\dblfloatsep{12\p@ \@plus 2\p@ \@minus 2\p@}
\setlength\dbltextfloatsep{24\p@ \@plus 2\p@ \@minus 4\p@}
%</10pt>
%<*11pt>
\setlength\dblfloatsep{12\p@ \@plus 2\p@ \@minus 2\p@}
\setlength\dbltextfloatsep{26\p@ \@plus 2\p@ \@minus 4\p@}
%</11pt>
%<*12pt>
\setlength\dblfloatsep{14\p@ \@plus 2\p@ \@minus 2\p@}
\setlength\dbltextfloatsep{28\p@ \@plus 2\p@ \@minus 4\p@}
%</12pt>
%    \end{macrocode}
% \end{macro}
% \end{macro}
%
% \begin{macro}{\@fptop}
% \begin{macro}{\@fpsep}
% \begin{macro}{\@fpbot}
%    The layout of floating objects on separate, `float' pages is controlled by
%    the parameters |\@fptop|, |\@fpsep| and |\@fpbot|. These parameters insert
%    stretchable amounts of white space between the top and the bottom of the  
%    page and the floats, while |\@fpsep| inserts white space between adjacent
%    floats. In order to ensure that any remaining space is filled, at least one
%    of |\@fptop| or |\@fpbot| has to contain stretchable space.
%    \begin{macrocode}
%<*10pt>
\setlength\@fptop{0\p@ \@plus 1fil}
\setlength\@fpsep{10\p@ \@plus 2fil}
\setlength\@fpbot{0\p@ \@plus 1fil}
%</10pt>
%<*11pt>
\setlength\@fptop{0\p@ \@plus 1fil}
\setlength\@fpsep{11\p@ \@plus 2fil}
\setlength\@fpbot{0\p@ \@plus 1fil}
%</11pt>
%<*12pt>
\setlength\@fptop{0\p@ \@plus 1fil}
\setlength\@fpsep{12\p@ \@plus 2fil}
\setlength\@fpbot{0\p@ \@plus 1fil}
%</12pt>
%    \end{macrocode}
% \end{macro}
% \end{macro}
% \end{macro}
%
% \begin{macro}{\@dblfptop}
% \begin{macro}{\@dblfpsep}
% \begin{macro}{\@dblfpbot}
%    The layout of floating objects in double-column mode on separate `float'
%    pages is handled in a similar way to that used for single-column mode
%    pages just described.
%    \begin{macrocode}
%<*10pt>
\setlength\@dblfptop{0\p@ \@plus 1fil}
\setlength\@dblfpsep{10\p@ \@plus 2fil}
\setlength\@dblfpbot{0\p@ \@plus 1fil}
%</10pt>
%<*11pt>
\setlength\@dblfptop{0\p@ \@plus 1fil}
\setlength\@dblfpsep{11\p@ \@plus 2fil}
\setlength\@dblfpbot{0\p@ \@plus 1fil}
%</11pt>
%<*12pt>
\setlength\@dblfptop{0\p@ \@plus 1fil}
\setlength\@dblfpsep{12\p@ \@plus 2fil}
\setlength\@dblfpbot{0\p@ \@plus 1fil}
%</12pt>
%    \end{macrocode}
% \end{macro}
% \end{macro}
% \end{macro}
%
%
% \subsection{Page styles}
%    A number of page styles are defined by the class file. Of course, more
%    elaborate styles can be designed using the |fancyheadings| package, but
%    once again, here we opt for the more restrained and discrete means of
%    guiding the reader. The page styles included in the standard \LaTeX\ book
%    class are either |headings|, |plain|, |empty| or |myheadings|. The standard
%    |headings| is here modified a little, so that the folios now appear 
%    centered at the foot of the page.
%
%    The page style \emph{style} is defined by defining the command |\ps@style|.
%    Definitions should be local only, and any stray spaces in definitions had 
%    better be avoided as they may turn up unexpectedly in the output.
%
% \begin{macro}{\@evenhead}
% \begin{macro}{\@oddhead}
% \begin{macro}{\@evenfoot}
% \begin{macro}{\@oddfoot}
%    The |\ps@style| commands define the macros |\@evenhead|, |\@evenfoot| and
%    their odd-page equivalents to define the running head and foot. The  
%    contents are produced inside a |\hbox| of width |\textwidth|.
% \end{macro}
% \end{macro}
% \end{macro}
% \end{macro}
%
% \subsubsection{Marking conventions}
%    The making of headings reflecting the sectioning commands, a page style
%    has to define commands such as |\chaptermark|, |\sectionmark| and so on
%    so that e.g., |\chapter| calls |\chaptermark{TEXT}|. These commands are
%    all defined with the help of the following macros, which have extended
%    in \LaTeX\ a bit as compared to the original \TeX\ |\mark| facility.
%    So we now have at our disposal:
%    \begin{description}
%    \item{|\markboth{Left}{Right}|} Adds both marks
%    \item{|\markright{Right}|} Adds a righthand-side mark
%    \item{|\leftmark|} used in the |\SIDEhead| and |\SIDEfoot| macros, 
%        getting the current lefthand-side mark
%    \item{|\rightmark|} also used in the |\SIDEhead| and |\SIDEfoot| 
%        macros, but getting the current righthand-side mark.
%    \end{description}
%
% \subsubsection{The headings and myheadings styles}
%    The two other styles, |plain| and |empty| are defined by \LaTeX.
%
% \begin{macro}{\ps@headings}
%    To begin with, the folios are centered at the foot of the pages, while
%    the running heads contain one of the marks. The contents of the running
%    head is determined both by the chapter and the section titles, hence we
%    let |\@mkboth| be |\markboth|.
%    \begin{macrocode}
%<*octavo>
\def\ps@headings{%
    \def\@oddfoot{\hfil\thepage\hfil}%
    \def\@evenfoot{\hfil\thepage\hfil}%
    \def\@evenhead{\hfil{\scshape\leftmark}\hfil}%
    \def\@oddhead{\hfil{\scshape\rightmark}\hfil}%
    \let\@mkboth\markboth
%    \end{macrocode}
% \end{macro}
%
% \begin{macro}{\chaptermark}
% \begin{macro}{\sectionmark}
%    As we use |\chaptermark| and |\sectionmark| to fill the running heads
%    we define these here. The use of |##1| as parameter of the commands, and
%    these will be defined when |\ps@headings| is executed.
%    \begin{macrocode}
\def\chaptermark##1{%
    \markboth {%
    \ifnum \c@secnumdepth >\m@ne
        \if@mainmatter
            \@chapapp\ \thechapter. \ %
        \fi
    \fi
    ##1}{}}%
\def\sectionmark##1{%
    \markright {%
    \ifnum \c@secnumdepth >\z@
        \thesection. \ %
    \fi
    ##1}}}
%    \end{macrocode}
% \end{macro}
% \end{macro}
%
% \begin{macro}{\ps@myheadings}
%    The definition of this page style is straightforward, as it is up to the
%    user to determine what goes into the running heads by setting the
%    |\markboth| and |\markright| commands in the text. We do, however, maintain
%    the presence of the centered folios at the foot of the pages. To ensure 
%    that the marking commands used by chapter and section commands are not 
%    used, we set them to a macro gobbling up the arguments.
%    \begin{macrocode}
\def\ps@myheadings{%
    \def\@oddfoot{\hfil\thepage\hfil}%
    \def\@evenfoot{\hfil\thepage\hfil}%
    \def\@evenhead{\hfil{\scshape\leftmark}\hfil}%
    \def\@oddhead{\hfil{\scshape\rightmark}\hfil}%
    \let\@mkboth\@gobbletwo
    \let\chaptermark\@gobble
    \let\sectionmark\@gobble}
%    \end{macrocode}
% \end{macro}
%
%
% \section{Document markup}
%
% \subsection{The title}
% \begin{macro}{\maketitle}
%    The three macros |\title|, |\author| and |\date| are part of \LaTeX,
%    and that info is therefore stored away in internal control sequences.
%    The |\date| macro gets today's date by default, so it may have to be
%    set explicitly to a particular date by the user.
%
%    If there is a request for a title, we will generate a simple, separate
%    titlepage. First, a local redefinition of |\footnotesize| and
%    |footnoterule|, so that we get a different appearance of the |\thanks|
%    command. These changes affect all footnotes, but since there really is
%    no need for true footnotes on a title page, this is hardly a problem.
%    \begin{macrocode}
\if@titlepage
    \newcommand\maketitle{\begin{titlepage}%
    \let\footnotesize\small
    \let\footnoterule\relax
    \let\footnote\thanks
%    \end{macrocode}
%    The title itself is centered vertically, with a little offset brought by
%    a |\vskip|.
%    \begin{macrocode}
    \null\vfil
    \vskip 60\p@
%    \end{macrocode}
%    Now we can set the title, using the |\LARGE| fount, followed by a little
%    bit of vertical space, and then the authors, set in |\large| fount. To
%    ensure everything lines up nicely in a single column, we resort to the
%    tabular environment. A little vertical space is then followed by the date.
%    \begin{macrocode}
    \begin{center}%
        {\LARGE \@title \par}%
        \vskip 3em%
        {\large
        \lineskip .75em%
        \begin{tabular}[t]{c}%
            \@author
        \end{tabular}\par}%
        \vskip 1.5em%
        {\large \@date \par}%
    \end{center}\par
%    \end{macrocode}
%    Calling |\@thanks| prints the information to go into the footnote, and
%    completes the printing on the page.
%    \begin{macrocode}
    \@thanks
    \vfil\null
    \end{titlepage}%
%    \end{macrocode}
%    Having finished the printing of the title page, we now clean up.
%    \begin{macrocode}
    \setcounter{footnote}{0}%
    \let\thanks\relax
    \let\maketitle\relax
    \gdef\@thanks{}
    \gdef\@author{}
    \gdef\@title{}}
\fi
%    \end{macrocode}
% \end{macro}
%
%
% \subsection{Chapters and section}
%    The coming definitions rely on two internal macros, |\@startsection| and
%    |\secdef|. Their syntax is simpe but a bit involved. The |\@startsection|
%    macro has 7 required arguments and 2 optional ones. The full set looks like
%    this:
%
%    |\@startsection|\meta{name}\meta{level}\meta{indent}\meta{beforeskip}
%    \meta{afterskip}\meta{style} \\ 
%    |[|*|] ||[|\meta{altheading}|]| \meta{heading}
%
%    \begin{description}
%    \item{\meta{name}} The name of the level command, e.g., `section'.
%    \item{\meta{level}} A number denoting the depth of the section,
%          chapter=1, section=2, etc.  A section number will be printed only
%          if \meta{level} is equal or smaller than the value of
%          \emph{secnumdepth}.
%    \item{\meta{indent}} The indentation of the heading from the left margin
%    \item{\meta{beforeskip}} The absolute value of this argument is the skip
%          to leave above the heading. If it is negative, then the paragraph
%          indent of the text following the heading is suppressed.
%    \item{\meta{afterskip}} If positive, it is the skip to leave below the
%          heading, else it is the skip to the right of a run-in heading.
%    \item{\meta{style}} Sets the style of the heading.
%    \item{$*$} When this is missing the heading is numbered and the
%          corresponding counter is incremented.
%    \item{\meta{altheading}} Gives an alternative heading to use in
%          the table of contents and in the running heads. This should
%          be present when the $*$ form is used.
%    \item{\meta{heading}} The heading of the new section.
%    \end{description}
%
%    The macro |\secdef| can be used when a sectioning command is
%    defined without using |\@startsection|. It has two arguments:
%
%    |\secdef|\meta{unstarcmds}\meta{starcmds}
%
%    \begin{description}
%    \item{\meta{unstarcmds}} Used for the normal form of the
%          sectioning command.
%    \item{\meta{starcmds}} Used for the $*$-form of the
%          sectioning command.
%    \end{description}
%
% \subsubsection{Definition of counters}
% \begin{macro}{\c@secnumdepth}
%    The \emph{secnumdepth} counter indicates to what level sectioning commands
%    have to make and include section numbers. More often than not, such numbers
%    are of little real value to the reader of a book: therefore, it is herein
%    set to a value that eliminates such number. The user can always bring them 
%    back in by resetting \emph{secnumdepth} in the preamble of the document.
%    \begin{macrocode}
\setcounter{secnumdepth}{-2}
%    \end{macrocode}
% \end{macro}
%
% \begin{macro}{\c@part}
% \begin{macro}{\c@chapter}
% \begin{macro}{\c@section}
% \begin{macro}{\c@subsection}
% \begin{macro}{\c@subsubsection}
% \begin{macro}{\c@paragraph}
% \begin{macro}{\c@subparagraph}
%    Nevertheless, all counters will be defined in case they are wanted.
%    The particular construction |\newcounter{\meta{newcount}}[\meta{oldcount}]|
%    defined \meta{newcount} as a counter which is reset every time
%    \meta{oldcount} is stepped. Of course, the counter \meta{oldcount} must  
%    have been defined.
%    \begin{macrocode}
\newcounter{part}
\newcounter{chapter}
\newcounter{section}[chapter]
\newcounter{subsection}[section]
\newcounter{subsubsection}[subsection]
\newcounter{paragraph}[subsubsection]
\newcounter{subparagraph}[paragraph]
%    \end{macrocode}
% \end{macro}
% \end{macro}
% \end{macro}
% \end{macro}
% \end{macro}
% \end{macro}
% \end{macro}
%
% \begin{macro}{\thepart}
% \begin{macro}{\thechapter}
% \begin{macro}{\thesection}
% \begin{macro}{\thesubsection}
% \begin{macro}{\thesubsubsection}
% \begin{macro}{\theparagraph}
% \begin{macro}{\thesubparagraph}
%    For each counter, the |\theCTR| macro defines how the counter CTR is 
%    printed.
%    \begin{macrocode}
\renewcommand\thepart{\@Roman\c@part}
\renewcommand\thechapter{\@arabic\c@chapter}
\renewcommand\thesection{\thechapter.\@arabic\c@section}
\renewcommand\thesubsection{\thesection.\@arabic\c@subsection}
\renewcommand\thesubsubsection{\thesubsection .\@arabic\c@subsubsection}
\renewcommand\theparagraph{\thesubsubsection.\@arabic\c@paragraph}
\renewcommand\thesubparagraph{\theparagraph.\@arabic\c@subparagraph}
%    \end{macrocode}
% \end{macro}
% \end{macro}
% \end{macro}
% \end{macro}
% \end{macro}
% \end{macro}
% \end{macro}
%
% \begin{macro}{\@chapapp}
%    The macro |\@chapapp| is defined to be |\chaptername|. But when the
%    |\appendix| command is issued, it is changed to |\appendixname|
%    \begin{macrocode}
\newcommand\@chapapp{\chaptername}
%    \end{macrocode}
% \end{macro}
%
% \subsubsection{Front, main and back matter}
%    Books are traditionally made up from three logical sections, known as
%    front matter, main matter and back matter. The actual \emph{text} of
%    a book is the main matter. The separation into different `matters' allows
%    slightly different ways of typesetting these sections.
%
% \begin{macro}{\frontmatter}
%    Front matter is typeset with roman page numbering and without numbers given
%    to any of its chapters.
%    \begin{macrocode}
\newcommand\frontmatter{\cleardoublepage
    \@mainmatterfalse\pagenumbering{roman}}
%    \end{macrocode}
% \end{macro}
%
% \begin{macro}{\mainmatter}
%    In main matter, we first clear the page, start page numbering with arabic
%    numerals and switch on chapter numbering (which will be visible only if
%    \emph{secnumdepth} has been set to a sufficiently high value, of course.
%    \begin{macrocode}
\newcommand\mainmatter{\cleardoublepage
    \@mainmattertrue\pagenumbering{arabic}}
%    \end{macrocode}
% \end{macro}
%
% \begin{macro}{\backmatter}
%    Back matter clears the page, turns off any chapter numbering, but leaves
%    the page numbering unchanged.
%    \begin{macrocode}
\newcommand\backmatter{\cleardoublepage
    \@mainmatterfalse}
%    \end{macrocode}
% \end{macro}
%
% \subsubsection{Parts}
%    The text of books may well be divided into a number of parts by the
%    writer(s). Here are the macros to deal with these major divisions of
%    a text, should they be required.
% \begin{macro}{\part}
%    A part is shown in the book by beginning on a new right-hand page and
%    we use a \emph{plain} pagestyle. 
%    \begin{macrocode} 
\newcommand\part{\cleardoublepage
    \thispagestyle{plain}%
%    \end{macrocode}
%    In case we are dealing with a double-column text, the page will be in a
%    single column but we have to remember we switch back to double-column mode.
%    \begin{macrocode}
    \if@twocolumn
        \onecolumn
        \@tempswatrue
    \else
        \@tempswafalse
    \fi
%    \end{macrocode}
%    To ensure that the fill glue does not disappear, we need an empty box.
%    \begin{macrocode}
    \hbox{}\vfil
%    \end{macrocode}
%    To make the actual heading, we issue a |\secdef| here.
%    \begin{macrocode}
    \secdef\@part\@spart}
%    \end{macrocode}
% \end{macro}
%
% \begin{macro}{\@part}
%    This is the macro which carries out the actual formatting of the title of
%    the part. If \emph{secnumdepth} equals -2 or larger, a number is given to
%    the part.
%    \begin{macrocode}
\def\@part[#1]#2{%
    \ifnum \c@secnumdepth >-2\relax
        \refstepcounter{part}%
        \addcontentsline{toc}{part}{\thepart\hspace{1em}#1}%
    \else
        \addcontentsline{toc}{part}{#1}%
    \fi
%    \end{macrocode}
%    Empty the mark registers, center the title on the page, and prevent any
%    breaking between lines, and finally reset the fount.
%    \begin{macrocode}
    \markboth{}{}%
    {\centering
    \interlinepenalty \@M
    \normalfont
%    \end{macrocode}
%    If we are dealing with a numbered part, print the number.
%    \begin{macrocode}
    \ifnum \c@secnumdepth >-2\relax
        \huge \partname~\thepart
        \par
%    \end{macrocode}
%    Leave some space before printing the title, and clear up with |\@endpart|.
%    \begin{macrocode}
        \vskip 20\p@
    \fi
    \Huge #2\par}%
    \@endpart}
%    \end{macrocode}
% \end{macro}
%
% \begin{macro}{\@spart}
%    A macro to deal with a part*. If this sectioning command was used, we do
%    not print a number, but otherwise the format is the same as for the
%    unstarred part.
%
%    \begin{macrocode}
\def\@spart#1{%
    {\centering
    \interlinepenalty \@M
    \normalfont
    \Huge #1\par}%
    \@endpart}
%    \end{macrocode}
% \end{macro}
%
% \begin{macro}{\@endpart}
%    A macro to finishes the part page. First, fill the current page,
%    \begin{macrocode}
\def\@endpart{%
    \vfil\newpage
%    \end{macrocode}
%    and, as we are dealing with a two-sided document, insert a blank page.
%    \begin{macrocode}
    \hbox{}%
    \thispagestyle{empty}%
    \newpage
%    \end{macrocode}
%    If we are dealing with a two-column document, switch back to double 
%    column mode.
%    \begin{macrocode}
    \if@tempswa
        \twocolumn
    \fi}
%    \end{macrocode}
% \end{macro}
%
% \subsubsection{Chapters}
% \begin{macro}{\chapter}
%    Chapters start on a new page, and the page style is set to \emph{plain}
%    \begin{macrocode}
\newcommand\chapter{%
    \if@openright\cleardoublepage
    \else\clearpage
    \fi
    \thispagestyle{plain}%
%    \end{macrocode}
%    Ensure that no float appears on the top of this page: it really would not
%    do to have something like a float appearing above a chapter title.
%    \begin{macrocode}
    \global\@topnum\z@
%    \end{macrocode}
%    Also, suppress any indentation of the first paragraph and issue the
%    |\secdef| which will take care of making the chapter title.
%    \begin{macrocode}
    \@afterindentfalse
    \secdef\@chapter\@schapter}
%    \end{macrocode}
% \end{macro}
%
% \begin{macro}{\@chapter}
%    The macro for numbered chapters. If \emph{secnumdepth} is larger than -1, 
%    and |\@mainmatter| is true, display the chapter number. Write a message to
%    the terminal to let the user know that a new chapter is about to be 
%    typeset.
%   \begin{macrocode}
\def\@chapter[#1]#2{%
    \ifnum \c@secnumdepth >\m@ne
        \if@mainmatter
            \refstepcounter{chapter}%
            \typeout{\@chapapp\space\thechapter.}%
            \addcontentsline{toc}{chapter}%
            {\protect\numberline{\thechapter}#1}%
        \else
            \addcontentsline{toc}{chapter}{#1}%
        \fi
    \else
        \addcontentsline{toc}{chapter}{#1}%
    \fi
%    \end{macrocode}
%    Having written the entry to the table of contents, store the (alternative)
%    title of the chapter with |\chaptermark|, add some white space to the lists
%    of figures and tables
%    \begin{macrocode}
    \chaptermark{#1}%
    \addtocontents{lof}{\protect\addvspace{10\p@}}%
    \addtocontents{lot}{\protect\addvspace{10\p@}}%
%    \end{macrocode}
%    Now call |\@makechapterhead| to typeset the actual chapter title. When 
%    dealing with text in double columns, this require a bit of attention as
%    we want the title to use the whole of the width of the text. 
%    The |\@afterheading| macro suppresses the indentation.
%    \begin{macrocode}
    \if@twocolumn
        \@topnewpage[\@makechapterhead{#2}]%
    \else
        \@makechapterhead{#2}%
        \@afterheading
    \fi}
%    \end{macrocode}
% \end{macro}
%
% \begin{macro}{\@makechapterhead}
%    The macro to typeset the actual chapter heading. Before we begin, bear in
%    mind that we have to retain the grid, so the total amount of space occupied
%    by the heading should be a whole number of lines. So, first some white 
%    space. Then open a group with no indenting, with ragged right text, and
%    reset the fount.
%    \begin{macrocode}
\def\@makechapterhead#1{%
    \vspace*{4\baselineskip}%
    {\parindent \z@ \raggedright \normalfont
%    \end{macrocode}
%    Find out if the chapter number has to be printed, and, if that is the case,
%    leave some white space between number and title.
%    \begin{macrocode}
    \ifnum \c@secnumdepth >\m@ne
        \if@mainmatter
            \Large \@chapapp{} \thechapter
            \par\nobreak
            \ifnum \@ptsize < 1
                \vskip 26\p@
            \else
                \vskip 22\p@
            \fi
        \fi
    \fi
%    \end{macrocode}
%    Set the title in Large fount, prevent a pagebreak from happening in the
%    middle or after the title, and leave a bit of white space before the text.
%    \begin{macrocode}
    \interlinepenalty\@M
    \Large #1\par\nobreak
    \vskip 34\p@ \@minus 4\p@
    }}
%    \end{macrocode}
% \end{macro}
%
% \begin{macro}{\@schapter}
%    The macro which takes care of unnumbered chapters, which is much more
%    straightforward, as it only has to typeset the chapter title.
%    \begin{macrocode}
\def\@schapter#1{%
    \if@twocolumn
        \@topnewpage[\@makeschapterhead{#1}]%
    \else
        \@makeschapterhead{#1}%
        \@afterheading
    \fi}
%    \end{macrocode}
% \end{macro}
%
% \begin{macro}{\@makeschapterhead}
%    The equivalent macro of |\@makechapterhead| but for typesetting the
%    heading of an unnumbered chapter.
%    \begin{macrocode}
\def\@makeschapterhead#1{%
    \vspace*{4\baselineskip}%
    {\parindent \z@ \raggedright
    \normalfont
    \interlinepenalty\@M
    \Large #1\par\nobreak
    \ifnum \@ptsize < 1
        \vskip 30\p@
    \else
        \vskip 34.8\p@
    \fi
    }}
%    \end{macrocode}
% \end{macro}
%
% \subsubsection{Lower-level headings}
% \begin{macro}{\section}
% \begin{macro}{\subsection}
% \begin{macro}{\subsubsection}
% \begin{macro}{\paragraph}
% \begin{macro}{\subparagraph}
%    All the lower-level headings are generated by using the |\@startsection|.
%    Because these various headings can occur anywhere on a page, some care
%    has gone into defining the vertical spacing surrounding the headings. The
%    reason for the different approach, compared to the standard book class, is
%    that we are trying our utmost to maintain the regularity of the printing
%    grid of text lines. Through the use of |\baselineskip| multiples and
%    fractions, the space given to headings and surrounding white space will
%    always be a multiple of print lines.
%    We take the opportunity here of toning down the `loudness' of some of the
%    headings compared to the standard book class.
%    \begin{macrocode}
\newcommand\section{%
    \@startsection{section}%
    {1}%
    {0em}%
    {-1.166\baselineskip \@minus -2\p@}%
    {0.835\baselineskip \@minus 2\p@}%
    {\centering\large\normalfont\scshape}}
\newcommand\subsection{%
    \@startsection{subsection}%
    {2}%
    {1em}%
    {-0.666\baselineskip \@minus -2\p@}%
    {0.333\baselineskip \@minus 2\p@}%
    {\normalfont\normalsize\scshape}}
\newcommand\subsubsection{%
    \@startsection{subsubsection}%
    {3}%
    {1em}%
    {-0.666\baselineskip \@minus -2\p@}%
    {0.333\baselineskip \@minus 2\p@}%
    {\normalfont\normalsize\itshape}}
\newcommand\paragraph{%
    \@startsection{paragraph}%
    {4}%
    {\z@}%
    {0\p@}%
    {-1em}%
    {\normalfont\normalsize}}
\newcommand\subparagraph{%
    \@startsection{subparagraph}%
    {5}%
    {\parindent}%
    {0\p@}%
    {-1em}%
    {\normalfont\normalsize}}
%    \end{macrocode}
% \end{macro}
% \end{macro}
% \end{macro}
% \end{macro}
% \end{macro}
%
% \subsection{Lists}
% \subsubsection{List parameters}
%    The type setting of the list environment is governed by a considerable
%    number of parameters, including |\leftmargin|, |\rightmargin|, 
%    |\listparindent|, |\itemindent|, |\labelwidth|, |\labelsep| and
%    |\partopsep|. By default, |\rightmargin|, |\listparindent| and
%    |itemindent| are set to |0pt|. For a list at level L, the command
%    |\@listL| is called, with L denoting `i', `ii', \ldots `vi'. By convention, 
%    |@listL| should set |\leftmargin| to |\leftmarginL|.
%
% \begin{macro}{\listmargin}
% \begin{macro}{\listmargini}
% \begin{macro}{\listmarginii}
% \begin{macro}{\listmarginiii}
% \begin{macro}{\listmarginiv}
% \begin{macro}{\listmarginv}
% \begin{macro}{\listmarginvi}
%    In double-column mode, margins are set somewhat smaller.
%    \begin{macrocode}
\if@twocolumn
    \setlength\leftmargini{2em}
\else
    \setlength\leftmargini{2.5em}
\fi
%    \end{macrocode}
%    \begin{macrocode}
\setlength\leftmargin{\leftmargini}
%    \end{macrocode}
% \end{macro}
% \end{macro}
% \end{macro}
% \end{macro}
% \end{macro}
% \end{macro}
% \end{macro}
%    The next three levels are set so that they are larger than the sum of
%    |\labelsep| and the width of the default labels, being `(m)', `vii.' and 
%    `M.'
%    \begin{macrocode}
\setlength\leftmarginii{2.2em}
\setlength\leftmarginiii{1.87em}
\setlength\leftmarginiv{1.7em}
\if@twocolumn
    \setlength\leftmarginv{.5em}
    \setlength\leftmarginvi{.5em}
\else
    \setlength\leftmarginv{1em}
    \setlength\leftmarginvi{1em}
\fi
%    \end{macrocode}
%
% \begin{macro}{\labelsep}
% \begin{macro}{\labelwidth}
%    The distance between label and text of an item is set by |\labelsep|, while
%    the width of the label is set by |\labelwidth|
%    \begin{macrocode}
\setlength\labelsep{.5em}
\setlength\labelwidth{\leftmargini}
\addtolength\labelwidth{-\labelsep}
%    \end{macrocode}
% \end{macro}
% \end{macro}
%
% \begin{macro}{\partopsep}
%    When a blank line is left before the list environment, we add an extra
%    vertical space of |\partopsep|, in addition to |\parskip| and |\topsep|.
%    Bearing in mind our attempts to maintain the regularity of the line grid,
%    the exact value of this space has to be set with some care.
%    \begin{macrocode}
%</octavo>
%<*10pt|11pt|12pt>
\setlength\partopsep{0.5\baselineskip \@plus 2\p@ \@minus 2\p@}
%</10pt|11pt|12pt>
%    \end{macrocode}
% \end{macro}
%
% \begin{macro}{\@beginparpenalty}
% \begin{macro}{\@endparpenalty}
% \begin{macro}{\@itempenalty}
%    These are the penalties incurred before and after a list or paragraph
%    environment, and set to a bonus value to encourage a page break at these
%    points.
%    \begin{macrocode}
%<*octavo>
\@beginparpenalty -\@lowpenalty
\@endparpenalty -\@lowpenalty
\@itempenalty -\@lowpenalty
%</octavo>
%    \end{macrocode}
% \end{macro}
% \end{macro}
% \end{macro}
%    
% \begin{macro}{\@listi}
% \begin{macro}{\@listI}
%    The macro |\@listi| sets the values of |\leftmargin|, |\parsep|, |\topsep|,
%    |\itemsep|, and so forth, for the lists that appear at top level. These
%    definitions are usually modified in function of the fount-size in use.
%    As we wish to maintain the printing grid, these adjustments are not used
%    here. Just to be on the safe side though, we define listI to hold a copy of
%    listi so that |\normalsize| can switch all parameters back.
%    \begin{macrocode}
%<*10pt|11pt|12pt>
\def\@listi{%
    \leftmargin\leftmargini
%    \end{macrocode}
%    \begin{macrocode}
%<*10pt>
    \parsep \z@
    \topsep 6\p@ \@plus 2\p@ \@minus 2\p@
    \itemsep \z@}
%</10pt>
%    \end{macrocode}
%    \begin{macrocode}
%<*11pt>
    \parsep \z@
    \topsep 6.6\p@ \@plus 2\p@ \@minus 2\p@
    \itemsep \z@}
%</11pt>
%    \end{macrocode}
%    \begin{macrocode}
%<*12pt>
    \parsep \z@
    \topsep 7.2\p@ \@plus 3\p@ \@minus 3\p@
    \itemsep \z@}
%</12pt>
%    \end{macrocode}
%    \begin{macrocode}
\let\@listI\@listi
\@listi
%    \end{macrocode}
% \end{macro}
% \end{macro}
%
% \begin{macro}{\@listii}
% \begin{macro}{\@listiii}
% \begin{macro}{\@listiv}
% \begin{macro}{\@listv}
% \begin{macro}{\@listvi}
%    Lists at higher levels need their own macros, and here they are. These do
%    not have a saved version, and are not modified by fount size commands.
%    \begin{macrocode}
\def\@listii{%
    \leftmargin\leftmarginii
    \labelwidth\leftmarginii
    \advance\labelwidth-\labelsep
    \itemsep \parsep}
\def\@listiii{%
    \leftmargin\leftmarginiii
    \labelwidth\leftmarginiii
    \advance\labelwidth-\labelsep
    \itemsep \parsep}
\def\@listiv{%
    \leftmargin\leftmarginiv
    \labelwidth\leftmarginiv
    \advance\labelwidth-\labelsep
    \itemsep \parsep}
\def\@listv{%
    \leftmargin\leftmarginv
    \labelwidth\leftmarginv
    \advance\labelwidth-\labelsep
    \itemsep \parsep}
\def\@listvi{%
    \leftmargin\leftmarginvi
    \labelwidth\leftmarginvi
    \advance\labelwidth-\labelsep
    \itemsep \parsep}
%</10pt|11pt|12pt>
%    \end{macrocode}
% \end{macro}
% \end{macro}
% \end{macro}
% \end{macro}
% \end{macro}
%
% \subsubsection{Enumerate}
% \begin{macro}{\theenumi}
% \begin{macro}{\theenumii}
% \begin{macro}{\theenumiii}
% \begin{macro}{\theenumiv}
%    The \emph{enumerate} environment uses four counters, where \emph{enumN}
%    controls the numbering at level N of the enumeration. These counters have
%    been defined by \LaTeX, but we use a different representation here.
%    \begin{macrocode}
%<*octavo>
\renewcommand\theenumi{\@arabic\c@enumi}
\renewcommand\theenumii{\@alph\c@enumii}
\renewcommand\theenumiii{\@roman\c@enumiii}
\renewcommand\theenumiv{\@Alph\c@enumiv}
%    \end{macrocode}
% \end{macro}
% \end{macro}
% \end{macro}
% \end{macro}
%
% \begin{macro}{\labelenumi}
% \begin{macro}{\labelenumii}
% \begin{macro}{\labelenumiii}
% \begin{macro}{\labelenumiv}
%    The label for each item in generated by these commands
%    \begin{macrocode}
\newcommand\labelenumi{\theenumi.}
\newcommand\labelenumii{(\theenumii)}
\newcommand\labelenumiii{\theenumiii.}
\newcommand\labelenumiv{\theenumiv.}
%    \end{macrocode}
% \end{macro}
% \end{macro}
% \end{macro}
% \end{macro}
%
% \begin{macro}{\p@enumii}
% \begin{macro}{\p@enumiii}
% \begin{macro}{\p@enumiv}
%    The expanding of |\p@enumN\theenumN| defines the output of a |\ref|
%    command when referencing an item at the Nth level in an enumerated list.
%    \begin{macrocode}
\renewcommand\p@enumii{\theenumi}
\renewcommand\p@enumiii{\theenumi(\theenumii)}
\renewcommand\p@enumiv{\p@enumiii\theenumiii}
%    \end{macrocode}
% \end{macro}
% \end{macro}
% \end{macro}
%
% \subsubsection{Itemise}
% \begin{macro}{\labelitemi}
% \begin{macro}{\labelitemii}
% \begin{macro}{\labelitemiii}
% \begin{macro}{\labelitemiv}
%    Itemising is governed by four commands, |\labelitemN|, defining the labels
%    at the various levels.
%    \begin{macrocode}
\newcommand\labelitemi{\normalfont\bfseries --}
\newcommand\labelitemii{$\m@th\cdot$}
\newcommand\labelitemiii{$\m@th\bullet$}
\newcommand\labelitemiv{$\m@th\ast$}
%    \end{macrocode}
% \end{macro}
% \end{macro}
% \end{macro}
% \end{macro}
%
% \subsubsection{Description}
% \begin{macro}{description}
%    The \emph{itemize} and \emph{enumerate} environments are defined by 
%    \LaTeX, but we have to define the description environment here.
%    \begin{macrocode}
\newenvironment{description}
    {\list{}%
    {\labelwidth\z@% 
    \itemindent-\leftmargin%
    \let\makelabel\descriptionlabel}}%
    {\endlist}
%    \end{macrocode}
% \end{macro}
%
% \begin{macro}{\descriptionlabel}
%    To change the format of the label, redefine |\descriptionlabel|
%    \begin{macrocode}
\newcommand*\descriptionlabel[1]{%
    \hspace\labelsep%
    \normalfont\bfseries #1}
%    \end{macrocode}
% \end{macro}
%
% \subsection{Defining new environments}
%     A number of separate environments for specific bodies of text can be
%     specified at class level. In the context of writing and making books,
%     here are a number of such environments.
%
% \subsubsection{Verse}
% \begin{macro}{verse}
%    A simple way of implementing the requirements of typesetting verse.
%    The environment is defined by tweaking some parameters of the list
%    environment. Each line of verse ends with a |\\|, and the macro takes
%    this to be |\@centercr|.
%    \begin{macrocode}
\newenvironment{verse}
    {\let\\=\@centercr
    \list{}{\itemsep \z@
    \itemindent -1.5em%
    \listparindent\itemindent
    \rightmargin \leftmargin
    \advance\leftmargin 1.5em}%
    \item[]}
    {\endlist}
%    \end{macrocode}
% \end{macro}
%
% \subsubsection{Quotation}
% \begin{macro}{quotation}
%    Also an implementation based on the list environment. Lines of the 
%    quotation are set more narrowly. 
%    \begin{macrocode}
\newenvironment{quotation}
    {\list{}{\listparindent 1.5em%
    \itemindent \listparindent
    \rightmargin \leftmargin
    \parsep \z@}%
    \item[]}
    {\endlist}
%    \end{macrocode}
% \end{macro}
%
% \subsubsection{Quote}
% \begin{macro}{quote}
%    For short quotations, the text is set as a quotation but without the
%    narrowing of the line length.
%    \begin{macrocode}
\newenvironment{quote}
    {\list{}%
    {\rightmargin\leftmargin}%
    \item[]}
    {\endlist}
%    \end{macrocode}
% \end{macro}
%
% \subsubsection{Titlepage}
% \begin{macro}{titlepage}
%    A titlepage starts and ends a page, and omits the folio. If the text is
%    typeset in double columns, this page alone reverts to single column mode.
%    \begin{macrocode}
\newenvironment{titlepage}{%
    \cleardoublepage
    \if@twocolumn
        \@restonecoltrue\onecolumn
    \else
        \@restonecolfalse\newpage
    \fi
    \thispagestyle{empty}}%
    {\if@restonecol
        \twocolumn
    \else
        \newpage
    \fi}
%    \end{macrocode}
% \end{macro}
%
% \subsubsection{Appendix}
% \begin{macro}{\appendix}
%    Typesetting appendices is somewhat different from the other chapters. We
%    have to reset section and subsection counters and redefine the macro
%    |\thesection| so that it produces an alphabetic enumeration of the 
%    appendices. We also set |\@chapapp| to |\appendixname|.
%    \begin{macrocode}
\newcommand\appendix{\par
    \setcounter{chapter}{0}%
    \setcounter{section}{0}%
    \renewcommand\@chapapp{\appendixname}%
    \renewcommand\thechapter{\@Alph\c@chapter}}
%    \end{macrocode}
% \end{macro}
%
% \subsection{Setting environment parameters}
% \begin{macro}{\arraycolsep}
%    Columns in an |array| environment are separated by 2 |\arraycolsep|
%    \begin{macrocode}
\setlength\arraycolsep{5\p@}
%    \end{macrocode}
% \end{macro}
%
% \begin{macro}{\tabcolsep}
%    Columns in a |tabular| environment are separated by 2 |\tabcolsep|
%    \begin{macrocode}
\setlength\tabcolsep{6\p@}
%    \end{macrocode}
% \end{macro}
%
% \begin{macro}{\arrayrulewidth}
%    The width of the lines in |array| and |tabular| environments
%    \begin{macrocode}
\setlength\arrayrulewidth{.4\p@}
%    \end{macrocode}
% \end{macro}
%
% \begin{macro}{\doublerulesep}
%    When rules are adjacent in |array| and |tabular| environments, this is
%    how far apart they are drawn
%    \begin{macrocode}
\setlength\doublerulesep{2\p@}
%    \end{macrocode}
% \end{macro}
%
% \begin{macro}{\tabbingsep}
%    Defines the space requested by the |\'| command in the text
%    \begin{macrocode}
\setlength\tabbingsep{\labelsep}
%    \end{macrocode}
% \end{macro}
%
% \begin{macro}{\@mpfootins}
%    The equivalent of |\skip\footins| for footnotes appearing in minipages.
%    \begin{macrocode}
\skip\@mpfootins = \skip\footins
%    \end{macrocode}
% \end{macro}
%
% \begin{macro}{\fboxsep}
% \begin{macro}{\fboxrule}
%    When using |\fbox| or |\framebox|, these macros define the amount of space
%    left between box and text, and the width of the rules in the box.
%    \begin{macrocode}
\setlength\fboxsep{3\p@}
\setlength\fboxrule{.4\p@}
%    \end{macrocode}
% \end{macro}
% \end{macro}
%
% \begin{macro}{\theequation}
%    Contrary to what is often seen in mathematical texts, here we number 
%    equations continuously throughout the text. Cross-referencing is just so
%    easy in TeX that there is not really any need for a different numbering 
%    system.
% \changes{1.1}{2003/10/10}{Equations numbered continuously}
%    \begin{macrocode}
\renewcommand\theequation{\@arabic\c@equation}
%    \end{macrocode}
% \end{macro}
%
%
% \subsection{Floating objects}
%    Floats cover a variety of object types. Here, we deal with two different
%    types: figure, and table. The typesetting of floats is of course governed 
%    by a number of specific macros.
%    \begin{description}
%    \item{|\fps@TYPE|} The default placement of TYPE floats
%    \item{|\ftype@TYPE|} The type number of floats or type TYPE. These are 
%        unique, powers of 2. For example, the type number of figures might be
%        1, tables 2, programs 4.
%    \item{|\ext@TYPE|} File extension of the file in which the contents list
%        of the TYPE floats is stored (example: |\ext@figure|=`lof'
%    \item{|fnum@TYPE|} The macro which generates the number for the caption of
%        the TYPE float.
%    \item{|\@makecaption|\meta{num}\meta{text}} The macro to make a caption 
%        with \meta{num} the value produced by |\fnum@TYPE| and \meta{text} the
%        text of the caption (in a |\parbox| of appropriate width). It is used
%        for all floating objects.
%    \end{description}
%
% \begin{macro}{\c@figure}
%    The format of the counter of figures. Often, the chapter number is part
%    of the number, as this supposedly makes it easier to find the figure again,
%    but that seems rather a poor justification, especially in the light of the
%    excellent capabilities of \TeX\ of making cross references.
%    \begin{macrocode}
\newcounter{figure}
\renewcommand\thefigure{\@arabic\c@figure}
%    \end{macrocode}
% \end{macro}
%
% \begin{macro}{\fps@figure}
% \begin{macro}{\ftype@figure}
% \begin{macro}{\ext@figure}
% \begin{macro}{\num@figure}
%    These are the parameters dealing with the `figure' floats.
%    \begin{macrocode}
\def\fps@figure{tbp}
\def\ftype@figure{1}
\def\ext@figure{lof}
\def\fnum@figure{\figurename~\thefigure}
%    \end{macrocode}
% \end{macro}
% \end{macro}
% \end{macro}
% \end{macro}
%
% \begin{macro}{figure}
% \begin{macro}{figure*}
%    Here we define the actual floating figure environments. The starred version
%    is used for double-column figures.
%    \begin{macrocode}
\newenvironment{figure}
    {\@float{figure}}
    {\end@float}
\newenvironment{figure*}
    {\@dblfloat{figure}}
    {\end@dblfloat}
%    \end{macrocode}
% \end{macro}
% \end{macro}
%
% \begin{macro}{\c@table}
%    And now for tables as floats. The numbering of tables follows the same
%    ideas as that of figures.
%    \begin{macrocode}
\newcounter{table}
\renewcommand\thetable{\@arabic\c@table}
%    \end{macrocode}
% \end{macro}
%
% \begin{macro}{\fps@table}
% \begin{macro}{\ftype@table}
% \begin{macro}{\ext@table}
% \begin{macro}{\num@table}
%    Define the parameters for table floats.
%    \begin{macrocode}
\def\fps@table{tbp}
\def\ftype@table{2}
\def\ext@table{lot}
\def\fnum@table{\tablename~\thetable}
%    \end{macrocode}
% \end{macro}
% \end{macro}
% \end{macro}
% \end{macro}
%
% \begin{macro}{table}
% \begin{macro}{table*}
%    Here we define the actual floating table environments. The starred version
%    is used for double-column tables.
%    \begin{macrocode}
\newenvironment{table}
    {\@float{table}}
    {\end@float}
\newenvironment{table*}
    {\@dblfloat{table}}
    {\end@dblfloat}
%    \end{macrocode}
% \end{macro}
% \end{macro}
%
% \begin{macro}{\@makecaption}
%    The |\caption| macro calls |\@makecaption| to typeset the caption of the
%    floating object. It uses the arguments \meta{number} and \meta{text}, so
%    that number and captiontext can be typeset. The macro assumes, rightly, 
%    that it is called inside a |\parbox| of appropriate width.
% \end{macro}
%
% \begin{macro}{\abovecaptionskip}
% \begin{macro}{\belowcaptionskip}
%    \begin{macrocode}
\newlength\abovecaptionskip
\newlength\belowcaptionskip
\setlength\abovecaptionskip{10\p@}
\setlength\belowcaptionskip{0\p@}
%    \end{macrocode}
%    The definition of the macro is |\long|, allowing more than a single
%    paragraph to be put into the caption.
%    \begin{macrocode}
\long\def\@makecaption#1#2{%
    \vskip\abovecaptionskip
%    \end{macrocode}
%    In order to see if the caption text fits on a single line, we typeset it
%    temporarily into a box.
%    \begin{macrocode}
    \sbox\@tempboxa{#1: #2}%
%    \end{macrocode}
%    Now measure the width of the box. If it exceeds the current |\hsize| then
%    the caption text has to be set as an ordinary paragraph.
%    \begin{macrocode}
    \ifdim \wd\@tempboxa >\hsize
        #1: #2\par
%    \end{macrocode}
%    If the text fits, center it on the line. This means using a |\hbox| in
%    vertical mode, hence the |\everypar| tokens are not executed. What could
%    be needed, if at all, is a resetting of the `minipage' flag, so we do that
%    here explicitly.
%    \begin{macrocode}
    \else
        \global \@minipagefalse
        \hb@xt@\hsize{\hfil\box\@tempboxa\hfil}%
    \fi
    \vskip\belowcaptionskip}
%    \end{macrocode}
% \end{macro}
% \end{macro}
%
%
% \section{Cross referencing}
%    Commands like |\section| or |\caption| write a contentsline to an
%    appropriate file. 
%    The line |\contentsline{section}{|\meta{title}|}{|\meta{page}|}| is 
%    written to the |.toc| file by the |\section| command. If sections
%    are numbered, the \meta{title} will show up as |\numberline{|\meta{num}
%    |}{|\meta{heading}|}|, the \meta{num} is the number generated by
%    |\thesection|. 
%
%    Similarly, the |\caption| command in a figure environment writes to a 
%    |.lof| file the line |\contentsline{figure}{\numberline{|\meta{num}|}{|
%    \meta{caption}|}}{|\meta{page}|}|.
%
%    The |\contentsline{|\meta{name}|}| command expands to |\l@|\meta{name}. 
%    Hence, we have to define |\l@chapter|, |\l@section| and so on, and for 
%    figures the similar |\l@figure|. These can be defined conveniently with the
%    command |\@dottedtocline{|\meta{level}|}{|\meta{indent}|}{|\meta{numwidth}
%    |}{|\meta{title}|}{|\meta{page}|}|.
%    \begin{description}
%    \item{\meta{level}} An entry is only produced if \meta{level} is smaller or
%        equal to the \emph{tocdepth} counter, with chapters being level 0, 
%        sections level 1 and so on.
%    \item{\meta{indent}} Defines the indentation from the outer left margin of
%        the contentsline.
%    \item{\meta{numwidth}} Defines the width of the box containing the section
%        number, if the \meta{title} includes a |\numberline|.
%    \end{description}
%
% \subsection{Table of contents, etc.}
% \begin{macro}{\@pnumwidth}
% \begin{macro}{\@tocrmargin}
% \begin{macro}{\@dotsep}
%    The three parameters control the appearance of the contents. Their 
%    definition through the |\newcommand| allows the use of the `em' measure,
%    meaning that the sizes defined can depend on the fount used. The role of
%    |\@pnumwidth| is to define the width of the box containing the page number.
%    |\@tocrmargin| defines the right margin for multi-line entries, obviously
%    it is a good idea ot have |\@tocrmarg| $\leq$ |\@pnumwidth|. Finally, the
%    separation between the dots on contentslines in mu units is given as a 
%    real number by |\@dotsep|. From a design perspective, these dots are very
%    unattractive, and an easy way of avoiding them appearing is to set the
%    separation to a very high number.
%    \begin{macrocode}
\newcommand\@pnumwidth{1.55em}
\newcommand\@tocrmarg{2.55em}
\newcommand\@dotsep{4500}
%    \end{macrocode}
% \end{macro}
% \end{macro}
% \end{macro}
%
% \begin{macro}{\tableofcontents}
%    This is the request for the production of a table of contents. Such a
%    table is always typeset as a single column, so it might be necessary
%    to switch temporarily to single-column mode.
%    \begin{macrocode}
\setcounter{tocdepth}{2}
\newcommand\tableofcontents{%
    \if@twocolumn
        \@restonecoltrue\onecolumn
    \else
        \@restonecolfalse
    \fi
%    \end{macrocode}
%    The title is set with |\chapter*|, to ensure that a possible running head
%    contains the correct information
%    \begin{macrocode}
    \chapter*{\contentsname
        \@mkboth{%
        \MakeUppercase\contentsname}
        {\MakeUppercase\contentsname}}%
%    \end{macrocode}
%    The actual table of contents itself is typeset by calling 
%    |\@starttoc{toc}|. If necessary, we also switch back to double-column mode.
%    \begin{macrocode}
    \@starttoc{toc}%
    \if@restonecol
        \twocolumn
    \fi}
%    \end{macrocode}
% \end{macro}
%
% \begin{macro}{\l@part}
%    The sectioning commands require additional macros to format the entries
%    in the table of contents. Of these various entries, the entry for parts
%    is somewhat more involved. First of all, ensure that if a pagebreak may
%    be imminent, it should occur before the entry of a part. The addition of
%    a little white space and the beginning of a group to keep changes local
%    occupies us here as well.
%    \begin{macrocode}
\newcommand*\l@part[2]{%
    \ifnum \c@tocdepth >-2\relax
    \addpenalty{-\@highpenalty}%
    \addvspace{2.25em \@plus\p@}%
    \begingroup
%    \end{macrocode}
%    Set |\parindent| to 0pt, and use |\rightskip| to reserve enough room for
%    the page numbers. The |\parfillskip| is made negative so that no overfull
%    box messages are issued.
%    \begin{macrocode}
        \setlength\@tempdima{3em}%
        \parindent \z@ \rightskip \@pnumwidth
        \parfillskip -\@pnumwidth
%    \end{macrocode}
%    Set the entry in large fount, and do a bit of housekeeping before 
%    finishing the macro. 
%    \begin{macrocode}
        {\leavevmode
        \large #1\hfil \hb@xt@\@pnumwidth{\hss #2}}\par
        \nobreak
        \global\@nobreaktrue
        \everypar{\global\@nobreakfalse\everypar{}}
    \endgroup
    \fi}
%    \end{macrocode}
% \end{macro}
%
% \begin{macro}{\l@chapter}
%    The entries of chapters is very similar to that of parts, and the 
%    definition of the macro follows the same lines.
%    \begin{macrocode}
\newcommand*\l@chapter[2]{%
    \ifnum \c@tocdepth >\m@ne
    \addpenalty{-\@highpenalty}%
    \vskip 1.0em \@plus\p@
%    \end{macrocode}
%    The |\numberline| macro needs the width of the box holding the number of
%    the chapter to be in the scratch |\@tempdima|. Also, begin a new group and
%    change some of the paragraph parameters.
%    \begin{macrocode}
    \setlength\@tempdima{1.5em}%
    \begingroup
        \parindent \z@ \rightskip \@pnumwidth
        \parfillskip -\@pnumwidth
        \leavevmode
%    \end{macrocode}
%    As we do no use |\numberline|, some manual adjustments are needed before
%    typesetting the entry. A pagebreak following a chapter entry should be
%    discouraged but not forbidden.
%    \begin{macrocode}
        \advance\leftskip\@tempdima
        \hskip -\leftskip
        #1\nobreak\hfil \nobreak\hb@xt@\@pnumwidth{\hss #2}\par
        \penalty\@highpenalty
    \endgroup
    \fi}
%    \end{macrocode}
% \end{macro}
%
% \begin{macro}{\l@section}
% \begin{macro}{\l@subsection}
% \begin{macro}{\l@subsubsection}
% \begin{macro}{\l@paragraph}
% \begin{macro}{\l@subparagraph}
%    The lower level entries are all governed by the |\@dottedtocline| macro.
%    \begin{macrocode}
\newcommand*\l@section{\@dottedtocline{1}{1.5em}{2.3em}}
\newcommand*\l@subsection{\@dottedtocline{2}{3.8em}{3.2em}}
\newcommand*\l@subsubsection{\@dottedtocline{3}{7.0em}{4.1em}}
\newcommand*\l@paragraph{\@dottedtocline{4}{10em}{5em}}
\newcommand*\l@subparagraph{\@dottedtocline{5}{12em}{6em}}
%    \end{macrocode}
% \end{macro}
% \end{macro}
% \end{macro}
% \end{macro}
% \end{macro}
%
% \begin{macro}{\listoffigures}
%    This is the request to produce a list of figures, and very similar in
%    construction to |\tableofcontents|.
%    \begin{macrocode}
\newcommand\listoffigures{%
    \if@twocolumn
        \@restonecoltrue\onecolumn
    \else
        \@restonecolfalse
    \fi
    \chapter*{\listfigurename
        \@mkboth{\MakeUppercase\listfigurename}%
        {\MakeUppercase\listfigurename}}%
        \@starttoc{lof}%
        \if@restonecol\twocolumn
    \fi}
%    \end{macrocode}
% \end{macro}
%
% \begin{macro}{\l@figure}
%    The macro which produces the actual entry in the list of figures.
%    \begin{macrocode}
\newcommand*\l@figure{\@dottedtocline{1}{1.5em}{2.3em}}
%    \end{macrocode}
% \end{macro}
%
% \begin{macro}{\listoftables}
%    The request to make a list of tables is given through this macro. Once
%    again, it is very similar to the |\tableofcontents| macro.
%    \begin{macrocode}
\newcommand\listoftables{%
    \if@twocolumn
        \@restonecoltrue\onecolumn
    \else
        \@restonecolfalse
    \fi
    \chapter*{\listtablename
        \@mkboth{%
        \MakeUppercase\listtablename}{\MakeUppercase\listtablename}}%
        \@starttoc{lot}%
        \if@restonecol\twocolumn
    \fi}
%    \end{macrocode}
% \end{macro}
%
% \begin{macro}{\l@table}
%    The macro which produces the actual entry in the list of tables.
%    \begin{macrocode}
\let\l@table\l@figure
%    \end{macrocode}
% \end{macro}
%
% \subsection{Bibliography}
% \begin{macro}{\bibindent}
%    The open bibliography format uses the indentation set by |\bibindent|.
%    \begin{macrocode}
\newdimen\bibindent
\bibindent=1.5em
%    \end{macrocode}
% \end{macro}
%
% \begin{macro}{thebibliography}
%    The environment is based on a generic list environment, using the 
%    \emph{enumiv} as internal counter to generate the labels of the list. If
%    and empty \emph{thebibliography} environment is found, issue a warning.
%    The environment executes a number of commands: |\sloppy| is used because
%    it is hard to do line breaks in bibliographies, |\sfcode`\.=1000\relax| is
%    issued so that a period does not produce and end-of-sequence space.
%    \begin{macrocode}
\newenvironment{thebibliography}[1]
    {\chapter*{\bibname
        \@mkboth{\MakeUppercase\bibname}
        {\MakeUppercase\bibname}}%
    \list{\@biblabel{\@arabic\c@enumiv}}%
        {\settowidth\labelwidth{\@biblabel{#1}}%
        \leftmargin\labelwidth
        \advance\leftmargin\labelsep
        \if@openbib
            \advance\leftmargin\bibindent
            \itemindent -\bibindent
            \listparindent \itemindent
            \parsep \z@
        \fi
        \usecounter{enumiv}%
        \let\p@enumiv\@empty
        \renewcommand\theenumiv{\@arabic\c@enumiv}}%
        \if@openbib
            \renewcommand\newblock{\par}%
        \else
            \renewcommand\newblock{\hskip .11em \@plus.33em \@minus.07em}%
        \fi
        \sloppy
%    \end{macrocode}
%    Reset parameters to their previous values before leaving the environment
%    \begin{macrocode}
        \clubpenalty4000
        \@clubpenalty \clubpenalty
        \widowpenalty4000
        \sfcode`\.=\@m}
        {\def\@noitemerr
        {\@latex@warning{Empty `thebibliography' environment}}%
    \endlist}
\newcommand\newblock{}
%    \end{macrocode}
% \end{macro}
%
% \subsection{The index}
% \begin{macro}{theindex}
%    The index environment procudes a two-column index, with each new entry 
%    having its own paragraph. In the text, the commands |\item|, |\subitem| and
%    |\subsubitem| are available to produce index entries at three levels. 
%    Each time a new letter of the alphabet is encountered, an |\indexspace|
%    amount of white space can be inserted.
%    \begin{macrocode}
\newenvironment{theindex}
    {\if@twocolumn
        \@restonecolfalse
    \else
        \@restonecoltrue
    \fi
    \columnseprule \z@
    \columnsep 35\p@
    \twocolumn[\@makeschapterhead{\indexname}]%
    \@mkboth{\MakeUppercase\indexname}%
    {\MakeUppercase\indexname}%
    \thispagestyle{plain}\parindent\z@
    \parskip\z@ \@plus .3\p@\relax
    \let\item\@idxitem}
%    \end{macrocode}
%    In case more text follows the index, and the document was typeset in
%    single-column mode, switch back to the one column mode before finishing
%    the index.
%    \begin{macrocode}
    {\if@restonecol
        \onecolumn
    \else
        \clearpage
    \fi}
%    \end{macrocode}
% \end{macro}
%
% \begin{macro}{\@idxitem}
% \begin{macro}{\subitem}
% \begin{macro}{\subsubitem}
%    The macros which typeset the actual index entries
%    \begin{macrocode}
\newcommand\@idxitem{\par\hangindent 40\p@}
\newcommand\subitem{\par\hangindent 40\p@ \hspace*{20\p@}}
\newcommand\subsubitem{\par\hangindent 40\p@ \hspace*{30\p@}}
%    \end{macrocode}
% \end{macro}
% \end{macro}
% \end{macro}
%
% \begin{macro}{\indexspace}
%    Define the amount of white space to insert between letter blocks in
%    the index.
%    \begin{macrocode}
\newcommand\indexspace{\par \vskip 10\p@ \@plus5\p@ \@minus3\p@\relax}
%    \end{macrocode}
% \end{macro}
%
%
% \subsection{Footnotes}
% \begin{macro}{\footnoterule}
%    Footnotes are often separated from the body of the text by a small rule.
%    To ensure that such a rule does not take up any vertical space, we have
%    to compensate for the height of the rule (of 0.4pt) by adding some vertical
%    skip. But we also have to avoid the rule from colliding with the footnote
%    text, so we add first a bit of negative skip, and make sure that we end up
%    eventually at the point where we began.
%    \begin{macrocode}
\renewcommand\footnoterule{%
  \kern-3\p@
  \hrule\@width.4\columnwidth
  \kern2.6\p@}
%    \end{macrocode}
% \end{macro}
%
% \begin{macro}{\c@footnote}
%    Footnotes are numbered within chapters, so we have to restart the numbering
%    when beginning a new chapter.
%    \begin{macrocode}
\@addtoreset{footnote}{chapter}
%    \end{macrocode}
% \end{macro}
%
% \begin{macro}{\@makefntext}
%    The macro which produces the footnote. We use a simple macro, in which the 
%    footnote text is typeset like a normal paragraph, without indentation 
%    except for its first line. The footnote markers printed in the text, which
%    refer to the actual footnotes are produced by the macro |\@makefnmark|. 
%    Here we take over the default definition without change. 
%    \begin{macrocode}
\newcommand\@makefntext[1]{%
    \parindent 1em%
    \noindent
    \hb@xt@1.8em{\hss\@makefnmark}#1}
%    \end{macrocode}
% \end{macro}
%
% \section{Initialisation}
% \subsection{Heading names}
%    The names of most of the headings can be (re)defined, and here we set
%    these names to their English equivalent. They can be renamed in the
%    text file through the |\renewcommand|.
%    \begin{macrocode}
\newcommand\contentsname{Contents}
\newcommand\listfigurename{List of Figures}
\newcommand\listtablename{List of Tables}
\newcommand\bibname{Bibliography}
\newcommand\indexname{Index}
\newcommand\figurename{Figure}
\newcommand\tablename{Table}
\newcommand\partname{Part}
\newcommand\chaptername{Chapter}
\newcommand\appendixname{Appendix}
%    \end{macrocode}
%
% \subsection{Date}
% \begin{macro}{\today}
%    The macro |\today| relies on the \TeX\ macros |\month|, |\day| and |year|
%    generate the date of the \LaTeX\ run. The macro has been re-arranged from
%    the standard \LaTeX\ format to conform to European usage, that is, 
%    day month year sequence.
%    \begin{macrocode}
\newcommand\today{\number\day \ \ifcase\month\or
  January\or February\or March\or April\or May\or June\or
  July\or August\or September\or October\or November\or December\fi
  \ \number\year}
%    \end{macrocode}
% \end{macro}
%
% \subsection{Double columns}
% \begin{macro}{\columnsep}
% \begin{macro}{\columnseprule}
%    The macros define the distance between the two columns, and the width of
%    rule separating the columns. Such a rule is usually unattractive and serves
%    no real purpose for the reader: it is therefore set to 0pt.
%    \begin{macrocode}
\setlength\columnsep{10\p@}
\setlength\columnseprule{0\p@}
%    \end{macrocode}
% \end{macro}
% \end{macro}
%
% \subsection{Page style}
%    Set the default page style to \emph{heading} and set the folio style to
%    \emph{arabic}
%    \begin{macrocode}
\pagestyle{headings}
\pagenumbering{arabic}
%    \end{macrocode}
%    If |twocolumn| was chosen as an option for typesetting the text, activate 
%    the double column mode with |\twocolumn|. Try to keep the columns balanced 
%    but do not make things too hard, hence issue a |\sloppy| command as well.
%    Normally, call the |\onecolumn| macro to typeset in a single column.
%    \begin{macrocode}
\if@twocolumn
  \twocolumn
  \sloppy
  \flushbottom
\else
  \onecolumn
\fi
%</octavo>
%    \end{macrocode}
%
% End of file `octavo.cls'.
% \Finale
% \PrintIndex
% \PrintChanges
\endinput
