% \iffalse meta-comment
%
% textsamples.dtx
%
% This file is part of the package textsamples for use with LaTeX2e
%
% Function: Access to utility macros and sample texts including lorem ipsum
% 
% Copyright (C) 2012,  Yannis Lazarides
%
% This program may be distributed and/or modified under the
% conditions of the LaTeX Project Public License, either version 1.2
% of this license or (at your option) any later version.
% The latest version of this license is in
%   http://www.latex-project.org/lppl.txt
% and version 1.2 or later is part of all distributions of LaTeX
% version 1999/12/01 or later.
%
% Please send error reports and suggestions for improvements to 
%    Yiannis Lazarides <y.lazarides@habtoorspecon.com>
% This work consists of the files listed in the README file.
%
%<*dtx>
\ProvidesFile{textsamples.dtx}
 [2012/02/13 v1.0 sample texts]  
%</dtx>
%% Now we tell TeX that we need LaTeX2e
%<package>\NeedsTeXFormat{LaTeX2e}
%<package>\ProvidesPackage{textsamples}[2012/02/13 v1.0 sample texts]
%<driver>\ProvidesFile{textsamples.drv}
% \fi
%
% \CheckSum{502}
%
%% \CharacterTable
%%  {Upper-case    \A\B\C\D\E\F\G\H\I\J\K\L\M\N\O\P\Q\R\S\T\U\V\W\X\Y\Z
%%   Lower-case    \a\b\c\d\e\f\g\h\i\j\k\l\m\n\o\p\q\r\s\t\u\v\w\x\y\z
%%   Digits        \0\1\2\3\4\5\6\7\8\9
%%   Exclamation   \!     Double quote  \"     Hash (number) \#
%%   Dollar        \$     Percent       \%     Ampersand     \&
%%   Acute accent  \'     Left paren    \(     Right paren   \)
%%   Asterisk      \*     Plus          \+     Comma         \,
%%   Minus         \-     Point         \.     Solidus       \/
%%   Colon         \:     Semicolon     \;     Less than     \<
%%   Equals        \=     Greater than  \>     Question mark \?
%%   Commercial at \@     Left bracket  \[     Backslash     \\
%%   Right bracket \]     Circumflex    \^     Underscore    \_
%%   Grave accent  \`     Left brace    \{     Vertical bar  \|
%%   Right brace   \}     Tilde         \~}
%%
% \GetFileInfo{textsamples.dtx}
% \iffalse
%<*driver>
\documentclass[a4paper,11pt]{ltxdoc}
\usepackage[latin1]{inputenc}
\usepackage[english]{babel}
\usepackage[T1]{fontenc}
\usepackage{soul}
\usepackage{hyperref}
\usepackage{color}
\definecolor{dred}{rgb}{.5,0,0}
\hypersetup{colorlinks=true,urlcolor=blue}
\makeatletter
\g@addto@macro{\MacroFont}{\footnotesize}
\usepackage[DIVcalc]{typearea}
\usepackage{xspace}
\makeatother
\EnableCrossrefs
\CodelineIndex
\RecordChanges
\begin{document}
   \DocInput{textsamples.dtx}
\end{document}
%</driver>
% \fi
% \newcommand\paket[1]{\textsf{#1}}
% \newcommand\option[1]{\textsf{#1}}
% \newcommand\name[1]{\caps{#1}}
% \title{\paket{textsamples} -- access  sample texts for testing of document classes, including paragrasphs of  \emph{Lorem ipsum}
% dummy text\thanks{Version: \fileversion}}
% \author{\name{Dr Yiannis Lazarides}\thanks{y.lazarides@habtoorspecon.com}}
% \date{\filedate}
% \changes{v1.0}{2004/05/15}{First release}
% \changes{v1.1}{2011/02/08}{Added support for typesetting dummy text without
% paragraphs in between.}
% \maketitle
% \begin{abstract}
%  A pot-pouri of small macros and useful utilities, to keep preambles short.
% \end{abstract}
% \tableofcontents
% \section*{Usage}
% To load the package specify 
% \begin{verbatim}
%     \usepackage{textsamples}
% \end{verbatim}
% \section{Motivation}
% I have been using these macros in a varity of documents and internal packages for a wide ranging 
% type of documents. Some of them have been developed as a result of question on tex.sx. A well
% developed document preparation system has to have well defined sectioning commands and a 
% set of ancillary macros that are semantically developed. Although this can aid is some future
% date to make \LaTeX\ more accessible to persons with disability, my primary focus is to aid
% the author with his writing and quest for better typography. A command defined semantically
% such as the |\emph| command provided by \LaTeX\ is easier to use in any document class. This
% package for example defineds |\BC| and |\AD|, where the European Union Style guide, demands that they 
% are typeset as small case. 
% in the preamble of your document. This package provides four
% macros. \DescribeMacro{\lipsum}The most important one is 
% \cmd{\lipsum}. This macro typesets the \emph{Lorem ipsum}
% paragraphs. It has an optional argument that allows one to specify the range
% of the  paragraphs. For example, \verb|\lipsum[4-57]| would typeset the paragraphs 4 to
% 57 and accordingly, \verb|\lipsum[23]| would typeset the 23\textsuperscript{rd}
% paragraph. Using \cmd{\lipsum} without its optional argument typesets
% the paragraphs specified by \DescribeMacro{\setlipsumdefault} 
% \cmd{\setlipsumdefault}. This is the second macro this package
% provides. By default it is set to |1-7|, resulting in a bit more
% than one page when used with |a4paper| and the standard or the
% \KOMAScript\ classes with default font settings. To change the
% default range use \verb|\setlipsumdefault{131-133}|. Of course, the numbers
% |131| and |133| are only examples that represent the first and the last
% paragraph selected to be typeset by default. 
%
% If used as explained above, the paragraphs generated by \verb|\lipsum| will be
% separated by the macro \verb|\par|, or, more precisely, every paragraph will be
% terminated by \verb|\par|. Sometimes, this may cause some unintended
% effects. Therefore the package provides the option \option{nopar} that
% causes \verb|\lipsum| to omit the terminating \verb|\par|. For this purpose,
% the package should be included via
% \begin{verbatim}
%     \usepackage[nopar]{lipsum}
% \end{verbatim}
%  The command \DescribeMacro{\lorem} is used to load a small paragraph of Lorem Ipsum text.
%  this is useful for testing new macros in MWE 
%
% \StopEventually{
%  \PrintChanges
%  \PrintIndex
% }
% \clearpage
%
% \section{Implementation}
%
% \DescribeMacro{\lorem}
% We sometimes need shorter paragraphs than those provided by the Lpack{lipsum}.
%    \begin{macrocode}
%<*package>
%    \end{macrocode}
%    \begin{macrocode}
\newcommand\lorem{Fusce adipiscing justo nec ante. Nullam in enim.
 Pellentesque felis orci, sagittis ac, malesuada et, facilisis in,
 ligula. Nunc non magna sit amet mi aliquam dictum. In mi. Curabitur
 sollicitudin justo sed quam et quadd. \par}
%    \end{macrocode}
% \begin{macro}{\fox}
% A common text used for testing fonts is:
% \begin{quote}
% ``The quick brown fox jumps over the lazy dog''
% \end{quote}
% we provide the command \cmd{\fox} for this.
%    \begin{macrocode}
\@ifundefined{fox}{%
    \newcommand{\fox}{``The quick brown fox jumps over the lazy dog''}}
    {\PackageWarning{textsamples}{Command fox has been defined}}
%    \end{macrocode}
% \end{macro}
%
% \begin{macro}{\dogs}
% This command is just more meaningless text. Although it provides some comfort
%  that an ``s'' can be important.
%    \begin{macrocode}
\newcommand{\dogs}{The sentence is often mistakenly rendered
as "The quick brown fox jumped over the lazy dog," which does
not include an ``s''. However, this can be corrected by
typing: "The quick brown fox jumped over the lazy dogs".}
%    \end{macrocode}
% \end{macro}
%
% \begin{macro}{\frogking}
%  We need a command to provide some text for testing hyphenation patterns.
%  \cmd{\frogking} from the book, is normally used for testing of hyphenation. 
%    \begin{macrocode}
\def\frogking{%
In olden times when wishing
still helped one, there lived a
king whose daughters were all
beautiful, but the youngest was so
beautiful that the sun itself,
which has seen so much, was
astonished whenever it shone in
her face. Close by the king's
castle lay a great dark forest,
and under an old lime-tree in the
forest was a well, and when
the day was very warm, the
king's child went out into the 
forest and sat down by the side
of the cool fountain, and when she 
was bored she took a golden ball, 
and threw it up on a high and caught it,
and this ball was her favorite plaything.}%
%    \end{macrocode}
% \end{macro}

% \begin{macro}{\onepar}
%  This command enforces the start of a paragraph
%    \begin{macrocode}
\def\onepar{\leavevmode 
In olden times when wishing
still helped one, there lived a
king whose daughters were all
beautiful, but the youngest was so
beautiful that the sun itself,
which has seen so much, was
astonished whenever it shone in
her face. Close by the king's
castle lay a great dark forest,
and under an old lime-tree in the
forest was a well, and when
the day was very warm, the
king's child went out into the 
forest and sat down by the side
of the cool fountain, and when she
was bored she took a golden ball, and threw 
it up on a high and caught it, and this
ball was her favorite plaything.\par}%
%    \end{macrocode}
% \end{macro}
% \subsection{Testing alphabets}
% For testing font alphabets we provide three commands:
% \begin{macro}{\ALPHABET}
%  This command, prints the alphabet. the alphabet is spaced out.
%  If the soul package is loaded, we space it out with the soul package.
%    \begin{macrocode}
\newcommand{\ALPHABET}{%
  A B C D E F G H I J K L M N O P Q R S T U V W X Y Z
}
%    \end{macrocode}
% \end{macro}
% \begin{macro}{\alphabet}
%    \begin{macrocode}
\newcommand{\alphabet}{a b c d e f g h i j k l m n o p q r s t u v w x y z}
%    \end{macrocode}
% \end{macro}
% \begin{macro}{\punctuation}
%   \begin{macrocode}
\newcommand{\punctuation}{+.;!-?/|}
%    \end{macrocode}
% \end{macro}
% 
% \begin{macro}{Alphabet}
% This macro typesets both the uppercase as well as the lowercase
% letters of the alphabet. It typesets them centered. Centering is within a group
% so as to avoid problems.
%    \begin{macrocode}
\newcommand{\Alphabet}{%
  \leavevmode
  \bgroup
  \centering
  \ALPHABET\par
  \alphabet\par
  \egroup
}
%    \end{macrocode}    
% \end{macro}
% We require the lipsum package to provide the text
%    \begin{macrocode}
\RequirePackage{lipsum}
%    \end{macrocode}
%
% 
% \begin{macro}{\loremlines}
% We sometimes need only a few linesof text, whereas the |lipsum| package
% provides paragraphs. \cmd{\loremlines}{ lines} provides exactly
% the number of lines specified.
% usage \cmd{\loremilines{10}}
% will typeset the first 10 lines from lipsum
%% Macro to type n lines from lipsum
% The command |\loremlines|\marg{n}
%    \begin{macrocode}
\newbox\one
\newbox\two
\long\def\loremlines#1{\setbox\one=\vbox {\lipsum}
\setbox\two=\vsplit\one to #1\baselineskip
\unvbox\two}
%    \end{macrocode}
%\end{macro}
% \section{Spacing commands}
% Here we collect all macros related to spacing. 
%
% \begin{macro}{\Z}
% We sacrifice the capital Z to use as a command for all spacing. where
% required. This is used extensively in formatting examples for Tables.
%    \begin{macrocode}
\newcommand\Z{\phantom{0}}
\newcommand\ZZ{\phantom{00}}
\newcommand\ZZZ{\phantom{000}}
\newcommand\ZZZZ{\phantom{0000}}
%    \end{macrocode}
% \end{macro}
% \section{Common abbreviations}
% The abbreviations package has a long list of abbreviations. However, we only need a few here,
% especially the common latin abbreviations. We need to handle spacing after the stops
% as well as ensure they do not 
% \begin{macro}{\hairsp}
% The command \cmd{\hairsp} is equivalent to 1pt. It cannot be used at the beginning
% of a paragraph. The \cmd{\hquad} is set at 0.5 em.
%    \begin{macrocode}
\newcommand{\hairsp}{\hspace{1pt}}% hair space
\newcommand{\hquad}{\hskip0.5em\relax}% half quad space   
%    \end{macrocode}
% \end{macro}
%
% \begin{macro}{\eg}
%    \begin{macrocode}
\RequirePackage{xspace}
\newcommand{\ie}{\textit{i.\hairsp{}e.}\xspace}
\newcommand{\eg}{\textit{e.\hairsp{}g.}\xspace}
\newcommand{\BC}[1]{\textsc{#1 bc}} %European Union Style Guide
\newcommand{\AD}[1]{\textsc{ad #1}} %European Union Style Guide
%    \end{macrocode}
% \end{macro}
% \begin{macro}{\otr}
% Shorthand for Output Routine. This is used extensively in the chapter for the
% output routine.
% \begin{macro}{\OTR} Typesets OTR
%    \begin{macrocode}
% Shorthands for typing
% OTR abbreviations
\def\otr{\textsc{OTR}\xspace}
\let\OTR\otr
%    \end{macrocode}
% \end{macro}
% \end{macro}
%
% \section{Shorthand Macros}
% A pot-pouri of shorthand macros. Most of them turned into macros to
% ensure typographical consistency.
%
% \begin{macro}{\fref}
% Shorthand for referencing figures and other common elements. This ensures that
% all referencing will be consistent in capitalization. In a future version I will provide
% a setting command.
%    \begin{macrocode}
\newcommand{\fref}[1]{Figure~\ref{#1}}
\newcommand{\tref}[1]{Table~\ref{#1}}
\newcommand{\eref}[1]{Equation~\ref{#1}}
\newcommand{\cref}[1]{Chapter~\ref{#1}}
\newcommand{\sref}[1]{Section~\ref{#1}}
\newcommand{\aref}[1]{Appendix~\ref{#1}}
%    \end{macrocode}
% \end{macro}
% We also provide shorthand commands for table filling
% \begin{macro}{\na}
% Typesets a em dash to represent \textit{not applicable}. This is in an effort to provide as many semantic
% commands as possible.
%    \begin{macrocode}
\newcommand{\na}{\quad--}% used in tables for N/A cells
%    \end{macrocode}
% \end{macro}
% \section{Commands for typesetting code}
% The most common requirement is to typeset the arguments
% of a macro and a backslash.
% 
% \begin{macro}{\env}
%    \begin{macrocode}
\newcommand*{\env}[1]{\texttt{#1}}
\newcommand*{\opt}[1]{\texttt{#1}}
\newcommand*{\meta}[1]{$\langle\textsl{#1}\rangle$}
\newcommand*{\marg}[1]{\texttt{\{}\meta{#1}\texttt{\}}}
\newcommand*{\oarg}[1]{\texttt{[}\meta{#1}\texttt{]}}
%    \end{macrocode}
%\end{macro}
%
%


% \section{Activities, key dates and editing commands}
% While writing, one needs to plan activities, mark todo items, create such lists and the like
% 
% 
% \begin{macro}{\keydate}
% The macro |\keydate|\marg{activity}\marg{date} writes all key dates in a file
% defined in |\keydate@file|\marg{filename}
% this can then be printed where is required.
%   |\out{Close Ceilings Rotana Floor 10}{Line 2}|
%   |\out{Rotana,  power-on}{12$^{th}$ Sept 2010}|
%   |\out{Rotana Qatar Cool switch-on}{12th September}|
%%% 
% |\chapter{Summary of  Key Dates}|
% |\immediate\closeout\tempfile|
% |\documentclass{article}
\usepackage{filecontents}
\begin{filecontents}{keydates.dat}

\end{filecontents}
\begin{document}
%% Define a new command for activity key-dates
 %% this can be saved for shipout later
 \newcommand{\keydate}[2]{#1  #2 \\}
 \newcommand{\out}[2]{%
   \immediate\write\tempfile{%
      \noexpand\keydate{ #1}{#2}}}

 %% We open the file to  to write the key-dates
 %% we will use it later to import
\newwrite\tempfile
\immediate\openout\tempfile=keydates.dat


 %% Example
 %% 

\out{Activity 1}{10 Sep 2012}
\out{Activity 2}{12 Sep 2012}
\out{Activity 3}{13 Sep 2012}

 %% finally we close the file
 %%
 \immediate\closeout\tempfile

 %% we now read the file and dostuff with it
 %% 
 \section{Summary of  Key Dates}
 \documentclass{article}
\usepackage{filecontents}
\begin{filecontents}{keydates.dat}

\end{filecontents}
\begin{document}
%% Define a new command for activity key-dates
 %% this can be saved for shipout later
 \newcommand{\keydate}[2]{#1  #2 \\}
 \newcommand{\out}[2]{%
   \immediate\write\tempfile{%
      \noexpand\keydate{ #1}{#2}}}

 %% We open the file to  to write the key-dates
 %% we will use it later to import
\newwrite\tempfile
\immediate\openout\tempfile=keydates.dat


 %% Example
 %% 

\out{Activity 1}{10 Sep 2012}
\out{Activity 2}{12 Sep 2012}
\out{Activity 3}{13 Sep 2012}

 %% finally we close the file
 %%
 \immediate\closeout\tempfile

 %% we now read the file and dostuff with it
 %% 
 \section{Summary of  Key Dates}
 \documentclass{article}
\usepackage{filecontents}
\begin{filecontents}{keydates.dat}

\end{filecontents}
\begin{document}
%% Define a new command for activity key-dates
 %% this can be saved for shipout later
 \newcommand{\keydate}[2]{#1  #2 \\}
 \newcommand{\out}[2]{%
   \immediate\write\tempfile{%
      \noexpand\keydate{ #1}{#2}}}

 %% We open the file to  to write the key-dates
 %% we will use it later to import
\newwrite\tempfile
\immediate\openout\tempfile=keydates.dat


 %% Example
 %% 

\out{Activity 1}{10 Sep 2012}
\out{Activity 2}{12 Sep 2012}
\out{Activity 3}{13 Sep 2012}

 %% finally we close the file
 %%
 \immediate\closeout\tempfile

 %% we now read the file and dostuff with it
 %% 
 \section{Summary of  Key Dates}
 \input{keydates.dat} 


\end{document}  


\end{document}  


\end{document} |
% 
\newwrite\tempfile
\immediate\openout\tempfile=keydates.tex
%    \begin{macrocode}
\newcommand{\keydate}[2]{#1  #2 \\}
\newcommand{\out}[2]{%
  {\index{key dates!#1}% add to index
\immediate\write\tempfile{\noexpand\keydate{ #1}{#2}}}}
\newcommand{\TODO}{\textcolor{red}{\bf TODO!}\xspace}
%    \end{macrocode}
% \end{macro}

% \section{Graphics macros}
% Most of these macros were developed for easier handling of 
% graphical input, they range from simple utilities to macros that could be
% packages of their own.
% \begin{macro}{\putgraphic}
%  The macro takes ones argument. used for image DB display
%    \begin{macrocode}
\long\def\putgraphic#1{%
\fbox{%
\begin{minipage}[b]{1.8cm}%

 \centering
 \vspace{3.8pt}\fbox{%
 \includegraphics[width=0.98\textwidth,height=2.3cm,keepaspectratio]{./images-01/#1}}%
  \vspace{0.2cm} #1%\captionof{figure}\relax
  \vspace{0.2cm}%
\end{minipage}}\hfil
}
%    \end{macrocode}
% \end{macro}
% \section{Charting macros}
% I must admit that I can never remember an adequate amount of TikZ keywords to
% partly because of their presentational aspect. The macros that follow are incomplete but provide
% some common Charts. I prefer to keep the data in separate files and  construct them with the
% minimal amount of code. For example this  provides a small bar plot that is common in reports that
% I produce.
% \begin{verbatim}
%%\begin{Chart}[purple]
%% \addTitle[Car Park Ventilation\newline Podia]{Jet Fans}
%% \def\dataTable{carparkventilation.dat}
%% \renderChart
%%\end{Chart}
% \end{verbatim}
% The data entry has the form:
% \begin{verbatim}
%\begin{filecontents}{carparkco.dat}
\begin{filecontents}{chapters.dat}
Label    value         num   other
{Chapter 1}         78           7   13
Chapter2         90           6   12
Chapter3         80           5    16
Chapter4         90           4    18
Chapter5         70           3    90
Chapter6         80           2    21
Chapter7         70           1    10
Chapter8         50          {}    {}
\end{filecontents}
%\end{filecontents}
% \end{verbatim}
% As you notice, we use the file contents package to write the data and hence we load it
% here for convenience. Another point to remember is to use brackets to enclose any labels
% or values that containing spaces. Similarly if there is no value to enter use a pair of braces.
%    \begin{macrocode}
 \usepackage{filecontents}
%    \end{macrocode}
% \begin{environment}{Chart}
% The environment |Chart| is used to plot progress charts. It is to be extended in a future
% version.
%    \begin{macrocode}
\newenvironment{Chart}[1][black!70!green]{%
%%  defaults
    \gdef\level##1{Level ##1}
    \def\setchartwidth##1{%
      \def\chartwidth{##1}}%
    \setchartwidth{3.9cm}%
    \def\chartcolor{#1}
    \newcommand\addTitle[2][test]{
%% For the chart title we set it in a minipage for
%% better control
    \def\charttitle{\minipage{4cm}%
       \footnotesize %
       \centering\textbf{##2}\\##1%
       \endminipage}}%
   \def\xlabel{Completion (\%)}%
%% renders the chart 
    \def\renderChart{%
%%
    \footnotesize%
%%
%%
    \IfFileExists{#1.dat}{Test}{}
   \begin{tikzpicture}
   \begin{axis}[
    xbar, width=\chartwidth,title=\charttitle,
    y=0.5cm, enlarge y limits={true, abs value=0.75},
    xmin=0, xmax=100,enlarge x limits={upper, value=0.25},
    xlabel=\xlabel,
    %ylabel=Label,
    xmajorgrids=true,
    ytick=data,
    yticklabels from table={\dataTable}{Label},
    nodes near coords, nodes near coords align=horizontal
     ]
    \addplot[draw=none, fill=\chartcolor] table [x=value, y=num]
    {\dataTable};
    \end{axis}%
    \end{tikzpicture}}}
{}
%    \end{macrocode}
% \end{environment}
%
% \section{Calculations}
% This section provides some macros for testing long integer typesetting,
% we use numprint for testing with the thousand separator as empty. This code was
% originally posted by me on text.sx. Usage |\Mersenne|\marg{number}
% \begin{macro}{\Mersenne}
% We use Mersene numbers to calculate a few numbers, this can be slow.
%    \begin{macrocode}
\usepackage{numprint}
\IfFileExists{bigintcalc}{\RequirePackage{bigintcalc}}{}
 \npthousandsep{ }
\def\Mersenne#1{%
  \begingroup
  \par\noindent\parindent=0pt
  $M_{#1}=$\par
  \def\exponent##1{\bigintcalcPow{2}{##1}}%
  \numprint{\bigintcalcSub{\exponent{#1}}{1}}
  \endgroup
}
%    \end{macrocode}
% \end{macro}
%
% 
%\section{Art Images}
% These macros have been developed specifically for art and photography books. You might as well
% look at the botticelli class if you want to use similar macros for a full class.
%There are many ways to print a small graphic or some text at a specific point on a page. The point can
%change position on odd or even pages. The packages |watermark| and |background| offer some sophisticated
%macros in this respect. Here we leverage the \LaTeX2e
% \begin{macro}{\samplepage}
%% Watermark package with small footprint
%% for more sophisticated macros use TikZ
%% Define a macro to print SAMPLE PAGE IN THE MARGIN
% The |changepackage| will be used to check for odd and even pages.
%    \begin{macrocode}
\IfFileExists{changepage.sty}{\RequirePackage{changepage}}{}
\IfFileExists{rotating.sty}{\RequirePackage{changepage}}{}

%    \end{macrocode}
%
% \begin{macro}{\even@samplepage}

% \begin{macro}{\odd@samplepage}
%    \begin{macrocode}
\def\even@samplepage{%
 \begin{picture}(0,0)
   \put(\Xeven,\Yeven){\turnbox{90}{\Huge \textcolor{\watermark@textcolor}{\watermark@text}}}
\end{picture}
}
%% Define a macro to print SAMPLE PAGE IN THE MARGIN
\def\odd@samplepage{%
 \begin{picture}(0,0)
   \put(\Xodd,\Yodd){\turnbox{90}{\Huge \textcolor{\watermark@textcolor}{\watermark@text}}}
 \end{picture}
}
%    \end{macrocode}
% \begin{macro}{watermarktext}
%  Define the watermark words
%    \begin{macrocode}
\def\watermarktext#1{\gdef\watermark@text{\fontfamily{phv}\selectfont#1}}
\def\watermarktextcolor#1{\gdef\watermark@textcolor{#1}}
\watermarktext{PRE-PRINT}
\watermarktextcolor{purple}
%    \end{macrocode}
% \end{macro}
%    \begin{macrocode}
%% redefine LaTeX's plain as myplain for headings
\def\ps@samplepage{\let\@mkboth\@gobbletwo
 \let\@oddhead\odd@samplepage\def\@oddfoot{\reset@font\hfil\thepage}
 \let\@evenhead\even@samplepage\def\@evenfoot{\reset@font\thepage\hfil}}
%%
%
%% We define two macros to position the watermark on the page
\def\Xodd{500}
\def\Xeven{-70}\def\Yeven{-810}
\def\Yeven{-\expandafter\strip@pt\textheight}
\let\Yodd\Yeven

\newlength\bleed@clearance
\setlength\bleed@clearance{3mm}
%% image corrections
%\the\dimexpr
%1in+\marginparwidth+\textwidth+\marginparsep+\evensidemargin-0.8in\relax
\newlength\oddcorrection
%\setlength\oddcorrection{\dimexpr(-1in-\oddsidemargin)+\bleed@clearance\relax}
\setlength\oddcorrection{\dimexpr(-1in+\oddsidemargin-\marginparsep+7pt)+\bleed@clearance\relax}

\newlength\evencorrection
\setlength\evencorrection{\dimexpr(-\marginparwidth-\marginparsep-\oddsidemargin-1pt)\relax}
%% caption related
\newlength\captiontextwidth
\setlength\captiontextwidth{\dimexpr 0.7\paperwidth\relax}
%% caption corrections
%\newlength\oddclearance
%\setlength\oddclearance{\dimexpr 7cm\relax}
%\newlength\evenclearance
%\setlength\evenclearance\bleed@clearance
%
%  we set two macros to cater for recto and verso
%  pages
%  \imagehskip is the shift of the image
%  we use changepage tofind out
%  if we recto or verso
%  \if@twoside is a switch for LaTeX standard classes
%  if not available we define
%  horizontal mode, get paranoid with spaces
\def\imagehskip{%
 \checkoddpage%
 \if@twoside% 
 \ifoddpage%
  \noindent\hspace*{\oddcorrection}%
\else%
  \hspace*{\evencorrection}%
\fi%
\else
   \noindent\hspace*{\oddcorrection}%
\fi
}%
% similarly define the captionskip amount. This also will
% depend on the paper setout as well as the page being
% recto or verso.
\def\captionhskip{%
\checkoddpage
\if@twoside%
 \ifoddpage%
%% Need to define a small margin at the end, if bleed is zero is a problem 50pt
  \hspace*{\dimexpr \oddcorrection-\bleed@clearance+0.3\paperwidth-50pt\relax}%
   \else%
   \hspace*{\dimexpr-1in-\oddsidemargin-\marginparsep\relax}%
 \fi%
 \else%
  \hspace*{\dimexpr 0.25\paperwidth+\oddcorrection-\bleed@clearance\relax}%
 \fi%
}%
%
%
%\pagebreak
%
%% we check for odd and even numbers with the changepage package.
%% we first check if the page is odd.
%
%% Include the graphic
%% The image needs to be the sixe of the paper - clearance
\def\addimage#1{%
\checkoddpage%
%% The image width is determined by the paper width - bleed clearance
\imagehskip\includegraphics[width=\dimexpr\paperwidth-\bleed@clearance\relax]{./graphics/#1}%
\par%
\vspace{0.5\baselineskip}%
}% do a vsplit here 
%
\newcommand\addcaption[2][Madonna.]{%
   \captionhskip
   \vbox\bgroup
    \hsize\captiontextwidth
    \parindent0pt\textbf{#1}\quad#2\par
    \egroup
    %\vfill\vfill
}%
%% The multicol caption, displays its caption as multicolumns
%%  it uses the package multicol with a default setting of 3 columns
%% 
\newcommand\addmulticolcaption[2][3]{%
\begin{multicols}{#1}
#2
\end{multicols}
\par
 } %
%
\pagestyle{samplepage}
%    \end{macrocode}
% \end{macro}
% \end{macro}
% \end{macro}
% \section{Custom files}
% It is quite likely that not everything will be covered by what is offered above. A custom file
% can be loaded if it exists. I normally use such files for throw-away definitions; that is macros
% that cannot be widely used. For example at work I have a file with name macros for my co-workers
% and people reporting to me. These names are used in activity reports and letters.
%    \begin{macrocode}
\IfFileExists{./Sections/definitions.tex}{\input{./Sections/definitions}}{}
%    \end{macrocode}
%    \begin{macrocode}
%</package>
%    \end{macrocode}
%
% \begin{thebibliography}{GMS94}
%
% \bibitem[GMS94]{GOOSSENS94}
% Michel Goossens, Frank Mittelbach, and Alexander Samarin.
% \newblock {\em The LaTeX Companion}.
% \newblock Addison-Wesley Publishing Company, 1994.
%
% \bibitem[vO96]{OOSTRUM96}
% Piet van Oostrum
% \newblock {\em Page layout in \LaTeX}, June 1996.
% \newblock (Available from CTAN as file \texttt{fancyhdr.tex}).
%
% \bibitem[SW94]{EBOOK}
% Douglas Schenck and Peter Wilson.
% \newblock {\em Information Modeling the EXPRESS Way}.
% \newblock Oxford University Press, 1994 (ISBN 0-19-508714-3).
%
% \bibitem[Thi99]{TTC199}
% Christina Thiele.
% \newblock \emph{The Treasure Chest: Package tours from CTAN},
% \newblock TUGboat, vol.~20, no.~1, pp~53--58, March 1999.
%
% \end{thebibliography}
%
%
% \Finale
% \PrintIndex
\endinput


\def\alicei{%	
  The King and Queen of Hearts were seated on their throne
  when they arrived, with a great crowd assembled about them
  --- all sorts of little birds and beasts, as well as the
  whole pack of cards: the Knave was standing before them,
  in chains, with a soldier on each side to guard him; and
  near the King was the White Rabbit, with a trumpet in one
  hand, and a scroll of parchment in the other.  In the very
  middle of the court was a table, with a large dish of
  tarts upon it: they looked so good, that it made Alice
  quite hungry to look at them --- ``I wish they'd get the
  trial done,'' she thought, ``and hand round the
  refreshments!''.  But there seemed to be no chance of this,
  so she began looking at everything about her to pass away
  the time.}%

\def\aliceii{%
  Alice had never been in a court of justice before, but she
  had read about them in books, and she was quite pleased to
  find that she knew the name of nearly everything there.
  ``That's the judge,'' she said to herself, ``because of his
  great wig.''.
  
  The judge, by the way, was the King, and as he wore his
  crown over the wig, (look at the frontispiece if you want
  to see how he did it,) he did not look at all comfortable,
  and it was certainly not becoming.
}

 \def\aliceiii{``And that's the jury-box,'' thought Alice, ``and those
  twelve creatures,'' (she was obliged to say ``creatures,''
  you see, because some of them were animals, and some were
  birds) ``I suppose they are the jurors.''.  She said this
  last word two or three times over to herself being rather
  proud of it: for she thought, and rightly too, that very
  few little girls of her age knew the meaning of it at all.
  However, ``jurymen'' would have done just as well.}

 \def\aliceiv{The twelve jurors were all writing very busily on slates.
  ``What are they doing?'' Alice whispered to the Gryphon.
  ``They can't have anything to put down yet, before the
  trial's begun.''.}
  
  \def\alicev{``They're putting down their names,'' the Gryphon
  whispered in reply, ``for fear they should forget them
  before the end of the trial.''.}
  
  \def\alicevi{``Stupid things!'' Alice began in a loud indignant voice,
  but she stopped herself hastily, for the White Rabbit
  cried out, ``Silence in the court!''; and the King put on
  his spectacles and looked anxiously round, to make out who
  was talking.\par}















